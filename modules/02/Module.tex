\documentclass[12pt,twoside]{article}
\usepackage[utf8]{inputenc}
\usepackage{rosoff}
\usepackage{fourier}
\usepackage[margin=0.7in]{geometry}
\usepackage{listings,booktabs,xparse,siunitx,parskip,xcolor}
\frenchspacing
\usepackage{graphicx}
\DeclareGraphicsExtensions{.jpg, .png}
\usepackage{hyperref}
\pagestyle{empty}
\title{Some disambiguation of various notational conventions}
\author{Math 352 Differential Equations}
\date{April 22, 2014}

\RenewDocumentCommand\vec{m}{\ensuremath{\mathbf{#1}}}

\begin{document}
\thispagestyle{empty}
\maketitle



There is a significant danger of confusion that arises when we attempt to port
our ideas about second-order linear equations to the case of a $2 \times 2$
system, as in \href{../../workshops/11/PhasePortraits.pdf}{Workshop 11}. The
danger is that an unwary or reckless reader may conflate the pair of functions
from Chapter 3 (an independent set of solutions of a second-order equation)
with the pair of functions from Chapter 7 (a single solution of the
\emph{system} of differential equations and its derivative).

When we discuss a system of differential equations $\vec{u}' = A\vec{u}$, the
understanding is that the  symbols $u_1$ and $u_2$ are reserved for the
entries of the vector $\vec{u}$. They cannot serve this function and also
retain their meaning from Chapter 3. These significations are incompatible.

The remainder of this note will explain the difference between the two pairs
and the planes they parameterize. From now on, let us consider the second-
order equation
\begin{equation} \label{eq:main}
    u'' + 4u' + 5u = 0.
\end{equation}
When we choose the fundamental system $u_1 = e^{-2t} \cos{t}$,
$u_2 = e^{-2t} \sin{t}$, we are also making a specific identification
of the \emph{solution set}
\begin{equation*}
    c_1 e^{-2t} \cos{t} + c_2 e^{-2t} \sin{t}
\end{equation*}
with the $(c_1, c_2)$-coordinate plane. If we were to choose a different 
fundamental system---say, 
\begin{align*}
    v_1 &= e^{-2t} (\cos{t} + \sin{t}) \\
    v_2 &= e^{-2t} (\cos{t} - \sin{t}) \quad \text{(observe that these
        functions have a nonzero Wronskian)},
\end{align*}
then the \emph{particular} solution
$2e^{-2t} \cos{t} - e^{-2t} \sin{t}$ is represented by a different coordinate
pair. It appears as the point $(2, -1)$ in the $(u_1,u_2)$-coordinate system
and as the point $(1/2,3/2)$ in the $(v_1,v_2)$-coordinate system.

Neither of these is the same thing as the phase plane, and it is essential
that you understand the difference well enough to distinguish them.

The phase plane arises when we consider a particular solution $U$ of the 
differential equation~\eqref{eq:main} and plot the trajectory of the
vector-valued function $\angl{U,U'}$. For example, we might have
\begin{equation*}
    \begin{pmatrix}
        U \\ U'
    \end{pmatrix} = \begin{pmatrix}
        2e^{-2t} \cos{t} - e^{-2t} \sin{t} \\
        -4 e^{-2t} \cos{t} - 3e^{-2t} \sin{t}
    \end{pmatrix}.
\end{equation*}
The point is, we are evidently not free to choose both entries independently.
Our choice of the first entry determines the second entry.

This is because the system of differential equations arising from Equation~\eqref{eq:main} is
\begin{align*}
    u'_1 &= u_2 \\ 
    u'_2 &= -5u_1 - 4u_2.
\end{align*}
It is not exactly convenient, but the most attractive solution is to write
\begin{align*}
    \vec{u}^{(1)} &= \begin{pmatrix}
        e^{-2t} \cos{t} \\ -e^{-2t} \cos{t} - 2e^{-2t} \sin{t}
    \end{pmatrix}, \\[2ex]
    \vec{u}^{(2)}&= \begin{pmatrix}
        e^{-2t} \sin{t} \\ 2 e^{-2t} \cos{t} - e^{-2t} \sin{t}
    \end{pmatrix}.
\end{align*}
The symbols are pronounced ``you-upper-one'' and ``you-upper-two'', respectively.
The parentheses in the superscripts are not pronounced and are only there to
help avoid confusion with exponents.

This allows us to stick with the irremediable convention that $u_1$ and $u_2$
denote the entries of a vector $\vec{u}$ while only slightly changing our
previous notation for the elements of a fundamental system of solutions.

That is, with this conventions, the old meaning of $c_1 u_1 + c_2 u_2$ is
\emph{the first entry} of the \emph{vector}
\begin{equation*}
    c_1 \vec{u}^{(1)} + c_2 \vec{u}^{(2)}.
\end{equation*}

If this last expression is our particular solution $\vec{u}$, then (putting it all 
together) we have
\begin{align*}
    \begin{pmatrix}
        u_1 \\ u_2
    \end{pmatrix} = \vec{u} &= c_1 \vec{u}^{(1)} + c_2 \vec{u}^{(2)} \\[2ex]
    &= \begin{pmatrix}
        c_1 e^{-2t} \cos{t} + c_2 e^{-2t} \sin{t} \\
        c_1 (-e^{-2t} \cos{t} - 2e^{-2t} \sin{t}) 
            + c_2 (2 e^{-2t} \cos{t} - e^{-2t} \sin{t}) 
    \end{pmatrix} \\[2ex]
    &= \begin{pmatrix}
        c_1 e^{-2t} \cos{t} + c_2 e^{-2t} \sin{t} \\
        (-c_1 + 2c_2) e^{-2t} \cos{t} + (-2c_1 -c_2) e^{-2t} \sin{t}
    \end{pmatrix}.
\end{align*}
Evidently, in the new context we obtain
\begin{align*}
    u_1 &= c_1 e^{-2t} \cos{t} + c_2 e^{-2t} \sin{t}, \\
    u_2 &= (-c_1 + 2c_2) e^{-2t} \cos{t} + (-2c_1 -c_2) e^{-2t} \sin{t}.
\end{align*}
\end{document}