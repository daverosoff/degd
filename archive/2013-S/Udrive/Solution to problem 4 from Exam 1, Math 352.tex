\documentclass[twocolumn,12pt]{article}
\usepackage[english]{babel}
\usepackage[utf8]{inputenc}
\usepackage[T1]{fontenc}
\usepackage[fontsize=12pt,baseline=14pt]{grid}
\usepackage[top=5.5cm,
  bottom=2.5cm,
  left=2.5cm,
  right=2.5cm,
]{geometry}
\usepackage{xparse}
\usepackage{fourier}
\usepackage{amsmath,amsthm}
\usepackage{paralist}
\usepackage{amssymb,latexsym}
%\input{commands}
\usepackage{MnSymbol}
\usepackage{gridleno}
\usepackage{hyperref}
\hypersetup{
  colorlinks=true,
  linkcolor=yotepurple,
  citecolor=grlnblue,
  urlcolor=grlngray}

\usetikzlibrary{calc,decorations.markings}

\NewDocumentCommand{\Z}{}{\mathbf{Z}}
\NewDocumentCommand{\C}{}{\mathbf{C}}
\NewDocumentCommand{\R}{}{\mathbf{R}}
\RenewDocumentCommand{\Re}{m}{\mathrm{Re}\;#1}
\RenewDocumentCommand{\Im}{m}{\mathrm{Im}\;#1}
%\NewDocumentCommand\arg{O{}}{\mathrm{arg}\;#1}
%\DeclareMathOperator{\arg}{arg}
\NewDocumentCommand\cis{O{}}{\mathrm{cis}\mkern2mu #1}
\NewDocumentCommand\Arg{O{}}{\mathrm{Arg}\mkern1mu #1}
\RenewDocumentCommand{\arg}{O{}O{}}{\mathrm{arg}_{#2}\mkern1mu #1}
\NewDocumentCommand\conj{m}{\overline{#1}}
%\DeclareMathOperator{\cis}{cis}
%\DeclareMathOperator{\Arg}{Arg}
\course{Differential Equations}
\courseid{352}
\professor{Dave Rosoff}
\term{Spring 2013}
\topic{Existence, uniqueness, and the zen of math}
\date{March 6, 2013 (Wed)}


\begin{document}

\makeheader

\begin{summary}
We review the meaning of the word ``solution'' in various contexts in order to discuss section 2.4 and the last problem from the March 4 exam. We conclude with what I hope are helpful tips for effective textbook reading.
\end{summary}

\section{What are solutions?}
There are two results in section 2.4 that tell us about when initial value problems have solutions. But before we discuss them, let's review what is meant by a ``solution'' to a differential equation and to an initial value problem. Pay attention to the words and how they are used. When you use them against convention and common scientific usage, you sound confused or ignorant at best and dangerously unhinged at worst. Based on the exam, more than half the class seems to be at least a little confused about what this means, so I suggest you don't write off this discussion.
\[
  y' + 2xy = 0
\]
This equation is grammatically correct no matter what function $y$ is. It is a sentence. Like English sentences, mathematical sentences can be grammatical without being true. For example, ``All College of Idaho courses are taught by unicorns.'' is a perfectly fine sentence, grammatically speaking. I hope its falsehood is equally evident. Some English sentences are grammatically valid, but seem to evade truth or falsehood. A famous example, due (I believe) to Chomsky, is ``Colorless green ideas sleep furiously.'' Mathematical sentences are more tractable. They are either true or false, if we are explicit enough about the context.

The above differential equation's mathematical correctitude depends on the choice of $y$. It is true for some $y$, but not for others. Moreover it is an equation of functions; it is not enough that for a particular $y$, there is one $x$-value for which it is true. We are looking for solutions that are ``valid'' in the sense of being mathematically correct for all $x$ in some \emph{interval}. Which interval that is cannot be made more precise at this point.

So, we are led to the following definition. We say that $y = \phi(x)$ is a solution to the differential equation if substitution of $\phi(x)$ for $y$ in said equation results in a mathematical correctitude for all $x$ in some interval $(\alpha, \beta)$. In general, there are lots of solutions to any particular differential equation.

When there is an initial condition, the meaning of ``solution'' changes. After all, it is another equation that must be satisfied. Most solutions to a differential equation are not going to be solutions to an associated initial value problem. Note that the meaning of the phrase ``initial value problem'' is ``differential equation plus initial condition''. The initial condition takes the form $y(x_0) = y_0$, where $x_0$ and $y_0$ are numbers.
A particular solution $y = \phi(x)$ \emph{to the differential equation} is a solution \emph{to the initial value problem} if, and only if,
\[
    \phi(x_0) = y_0.
\]

\section{Existence and Uniqueness}
The first theorem concerns \emph{linear differential equations}. Before you go on, you need to make sure you understand what this phrase means. It was quite evident from the exam that many of you do not. That isn't a value judgment, only a statement of fact. I do not think it is a difficult concept, but you do have to do some work to understand what the definition is saying. When you encounter new terms, it is your responsibility to make sure you understand them and to \emph{ask questions} if you don't, or aren't sure you do. Now the definition of a linear DE is such that there are natural functions associated to it. These functions are called the coefficients in the differential equation. In our textbook, they are called $p(x)$ and $g(x)$. Theorem 2.4.1 states that if these functions $p$ and $g$ are continuous on the interval $( \alpha, \beta)$, then \emph{every initial value problem}
\[
    y' + p(x)y = g(x), \quad y(x_0) = y_0
\]
\emph{has a unique solution} (in the precise sense outlined above) provided that $\alpha < x_0 < \beta$, and moreover that this solution is ``valid'' throughout the interval $( \alpha, \beta)$. 

I observed widespread confusion, among those who referred to this theorem, about the meaning of the continuity hypothesis. This hypothesis does not refer to the solution of the initial value problem. That function's continuity is asserted in the \emph{conclusion} of the theorem. Rather, it refers to the coefficient functions $p$ and $g$. This leads me to another important point. Even when you are pressed for time, you must be precise in your thinking and writing. Context-free references to ``it'' or ``the function'' are so vague as to be almost meaningless. If what you mean is that $p$ and $g$ are continuous, say \emph{that}, and not some elliptical or cryptic paraphrase. 

To sum up: the theorem states that if $p$ and $g$ are continuous on some interval, then
\begin{inparaenum}[(a)]
  \item each initial value problem $y' + p(x)y = g(x)$, $y(x_0) = y_0$ has a unique solution and
  \item said solution is differentiable on the same interval.
\end{inparaenum}

The situation is different for equations that are nonlinear. Many of you seem to believe that initial value problems for nonlinear equations usually do not have unique solutions. I believe that part of the confusion stems from conflation of the ideas ``solution to a differential equation'' and ``solution to an initial value problem''. Solutions to differential equations are hardly ever unique. Solutions to initial value problems almost always are (problem 26 from 2.4, on the weekly homework, is an illustration of how uniqueness can fail).

Theorem 2.4.2 is similar to 2.4.1. There are continuity hypotheses that have to be checked. Here, though, there are no coefficient functions. Rather, we have an initial value problem in the form
\[
    y' = f(x, y), \quad y(x_0) = y_0.
\]
There are no $p$ or $g$ in sight. The continuity hypotheses are about $f$. Since $f$ is a function of two variables, we do not discuss its continuity at a \emph{number}, but at a \emph{point} in the plane---namely $(x_0, y_0)$. The theorem states that if both $f$ and $\partial f/\partial y$ are continuous at $(x_0, y_0)$, then the IVP has a unique solution on some interval containing $x_0$. Unlike in the linear case, we have no control over the size of this interval---the domain of the solution. But that is the only difference. If the continuity hypotheses are satisfied, \emph{the solution is unique}. There is only one function $y = \phi(x)$ that ``works''.

\section{The exam problem}
In problem 4 on Exam 1, you were presented with two initial value problems, one linear and one nonlinear. For the linear problem, a full-credit response would have cited
\begin{inparaenum}[(a)]
    \item linearity of the differential equation;
    \item continuity of the coefficient functions \emph{on the entire real line}, that is, for $-\infty < x < \infty$;
    \item Theorem 2.4.1, which guarantees the existence of a unique solution in the presence of the above;
    \item the validity of the solution on the interval where the coefficients are continuous, namely $\R = (-\infty, \infty)$,
\end{inparaenum}
and would have done all this in coherent, precise sentences. 

A full-credit response to the nonlinear problem would have been similar, but with appropriate modifications of the continuity conditions and the extent of the interval of validity.

\section{Conclusion: how to read}
Mathematical language is dense and hard to understand because mathematical ideas are more subtle and nuanced than any everyday concepts. Such nuances require precise and sometimes difficult language in order to be clearly delineated. Therefore, \emph{it is not possible to understand mathematics without understanding its language}. Those who think it is are conflating mathematics with computation. Computation is an important part of mathematics, but it is far from the whole story. Here are some tips to help you learn to internalize this language and make it your own. 
\begin{itemize}
    \item \emph{Take copious notes.} When I am reading a new mathematical text, I paraphrase almost everything as I read, either by hand or digitally. I end up with what amounts to a copy of the book I've just read, in my own hand. It is impossible for me to explain why this is effective, but every mathematician I know does it. The act of rephrasing and explaining the words clarifies the concepts at issue and focuses your attention on the places where your understanding is lacking. I also never read about 90\% of the notes I take in this way. Once I understand the material, they have served their purpose. The exceptions are notes on difficult calculations or other things I have trouble remembering and am likely to wish to refer to again.
    \item \emph{See periods as invitations to stop and think.} Math books are not like other books. You need to read and read and reread until you get it. Many people like to bounce back and forth between the problems and the text. I hope it is clear from the exam that reading the text is no longer optional at this point in your mathematical training. Many students resist reading, because it sucks until you get used to it\footnote{You will get used to it, if you practice. I promise.}. You have to retrain yourself to slow down and think deeply about every sentence.
    \item \emph{Think of questions to ask in class.} Class improves as more people talk in it. I know it's tough first thing in the morning, but try to make the effort. People who are trying to learn like it when you ask questions, especially your classmates who are less inclined to call attention to themselves. Still you should be decorous and refrain from dominating the conversation. The best opportunity to think of questions is---surprise!---when you are reading the textbook or working daily homework problems!
    \item \emph{Read every day.} ``What a magician is the subconscious! If only it would work regular hours.''\footnote{R.\ Chandler, \emph{The Long Goodbye}.} You need to give your brain a little math food every day, even if you don't have time to read much or work any problems. Some people like to read a bit before bed. You don't have to read a whole lot, necessarily, although there are always deadlines to cope with. But it's essential to keep your subconscious chugging away on these things.
    \item \emph{Do not rely too much on others.} Part of what makes mathematics difficult is that it can be isolating at times. Eventually, you will have to grapple with these ideas alone, at least if you want to really understand them. Talking with other people, especially other people who are learning with you, is absolutely invaluable. But it cannot substitute for the hard work you must put in alone, just you and your lovely smart wrinkly brain.
    \item \emph{Computation is only part of the story.} The computations that we do in class are there as aids to understanding the mathematics behind them, not as an end in themselves. In the modern age all differential equations of interest are solved numerically with computers. But computers cannot yet understand math for you. That's a big part of why we still compute by hand. It aids in deeper understanding.
    \item \emph{Learn to be comfortable with ignorance and confusion.} This is the hardest and most important piece of advice I have for you. Most people who get as far in math as you do\footnote{The College graduates about 250 students every year. Fewer than 40 take differential equations. So you are in the 84th percentile of C of I students math-wise.} are used to being good at it. They do not like feeling as though nothing makes sense, or even feeling as though one thing doesn't make sense. We all must retrain ourselves not to recoil from these emotions. After all, what manner of scholars believe they already know everything there is to know? 
    \item \emph{Confusion and ignorance are the default state in mathematics.} Accustom yourself to them; entertain them in your home; become their intimate friend. Perplexity is noble: it can only be attained by those with the determination and strength to expand their own intellectual limits. 
    \item \emph{Being wrong is a valuable life skill.} It's true. And most people are absolutely awful at it. Studying math makes you a better person because it helps you keep perspective and stay humble. As strange as it seems, your goals should include becoming really, \emph{really}, \textsc{really} good at being wrong: so good at it that you don't mind it or fear it at all. Welcome your moments of incorrectitude as the approach of an old friend. Those are the moments when you have a chance to learn something.
    \item \emph{It is possible to be comfortable in confusion.} As you practice being wrong, learn to recognize the mental states that accompany it. We call them \emph{confusion} and \emph{ignorance}. Become comfortable enough that you can shrug off anxiety and fear of ignorance, and your mind will relax and be open; then you can truly listen when you are reading the book, solving a problem, or talking with friends; then your mind has found itself in a really creative state; then true learning begins, and true mastery awaits you.
\end{itemize}
None of this is easy; but all of it is possible, if you have the will to attempt it.
\end{document}