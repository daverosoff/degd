\documentclass[12pt]{exam}
% \usepackage{pslatex}
\usepackage[utf8]{inputenc}
\usepackage[T1]{fontenc}
\usepackage{xparse}
%\usepackage{graphicx}
%\DeclareGraphicsExtensions{.jpg, .png}
\usepackage{amsmath}
%\usepackage{amsfonts}
\usepackage{fourier,MnSymbol}
\usepackage{enumerate}
\usepackage{siunitx}
\usepackage{tikz}
\usetikzlibrary{arrows}
\NewDocumentCommand\N{}{\mathbf{N}}
\NewDocumentCommand\R{}{\mathbf{R}}
\NewDocumentCommand\Z{}{\mathbf{Z}}
\NewDocumentCommand\Q{}{\mathbf{Q}}
\NewDocumentCommand\C{}{\mathbf{C}}
\NewDocumentCommand\Arg{m}{\mathrm{Arg}\;#1}
\RenewDocumentCommand\Re{m}{\mathrm{Re}\;#1}
\RenewDocumentCommand\Im{m}{\mathrm{Im}\;#1}
\NewDocumentCommand\dwrspace{m}{\vspace*{\stretch{#1}}}

%%%%%%%%%%%%%%%%%%%%%%%%%%%%%%%%%%%%%%%%%%%%%%%%%%%%%%%%%
% Fill in these!!

\NewDocumentCommand\termID{}{Spring 2013} % FILL IN TERM ID eg `Spring 2012'
\NewDocumentCommand\courseID{}{Mathematics 352} % FILL IN COURSE ID eg `Mathematics 111'
\NewDocumentCommand\assignmentID{}{Exam 3} % FILL IN ASSIGNMENT ID eg `Quiz 3'
\NewDocumentCommand\dateID{}{April 29, 2013} % FILL IN DATE

%%%%%%%%%%%%%%%%%%%%%%%%%%%%%%%%%%%%%%%%%%%%%%%%%%%%%%%%%

\firstpageheader{}{}{}
\runningheader{\textbf{\termID}}
 {}
 {\textbf{\courseID}}
 %{\emph{Page \thepage~of \numpages}}
\runningheadrule
\setlength{\parskip}{1ex}
\setlength{\parindent}{0pt}
\pagestyle{head}

\begin{document}
\noindent
\textbf{{\large \courseID \\ \assignmentID}}

\noindent
\dateID; 60 minutes  \hfill Name: \underline{\hspace{3in}} 
% \hfill Name: \underline{\hspace{0.5in}Answers\hspace{2in}}

% \printanswers
\addpoints

\noindent
This exam is closed book; you can use a calculator (\emph{not} a mobile phone) but no other electronic aids or printed references. \emph{If the wording or intent of any question is unclear, please ask me to clarify.} I am not trying to confuse you with the problem statements.

You can use your own paper or the provided blank copy paper. \textsc{Do not write anything you want graded on the exam paper.} Please write your name on each page you hand in. Show all your reasoning and all pertinent calculations. \emph{Give all answers in exact form. Decimal approximations of any accuracy will not receive full credit.}

When you have finished the exam, place this cover sheet on top of it and fold the packet in half \emph{the long way, with your name facing out}.

\dwrspace{1}

\begin{center}
\textsc{You are strongly encouraged to read all the problems before beginning.}
\end{center}

\dwrspace{1}

\begin{figure}[h]
\centering
    \begin{tikzpicture}
        \path[draw] (0, 0) rectangle (2.6, 4);
        \path[draw] (0, 4) -- (2.0, 4.7) -- (2.0, 4);
    \end{tikzpicture}
\end{figure}
\dwrspace{1}

\begin{center}
    \gradetable
\end{center}

\dwrspace{1}

\begin{center}
    {\Large \emph{Good luck!}}
\end{center}

\newpage

\begin{center}
    {\Large \textsc{No Work On This Page}}

    {\Large \textsc{All Work on Separate Pages}}
\end{center}

\begin{questions}  

%%%%%%%%%% From SP 10? 11?
\question Consider a mass-spring system. Suppose it is perturbed and released in such a way that the displacement function for the mass is the solution of the initial value problem
\begin{displaymath}
    2u'' + 0.6u' + 0.45 u = 0, \quad u(0) = 3, \quad u'(0) = 0.
\end{displaymath}
Solve the initial value problem to find $u(t)$.

\dwrspace{1}

\question
\emph{In this problem, assume that the acceleration $g$ due to gravity is exactly $\SI{10}{m/s^2}$.}

\begin{parts}
    \part Consider a spring-mass system with a mass of $\SI{6}{\kilo\gram}$ that stretches the spring a length of $\SI{5/3}{\meter}$ when first attached. Suppose this system operates in a viscous medium that provides $\SI{60}{\newton}$ of resistive force to a mass moving at $\SI{2}{m/s}$. 

    Find the position function for the motion that results if the mass is moved $\SI{2}{\meter}$ below its equilibrium position and then released. Show all work. Classify the motion as overdamped, underdamped, or critically damped.

    \part How should the damping constant $\gamma$ be changed to obtain a critically damped system, if the mass and spring constant are left unchanged? (Give an explicit value for $\gamma$.)
\end{parts}

\dwrspace{1}

\question Use whatever method you like to find the general solution to the equation
\[
    y'' - 10y' + 25y = e^{5t}.
\]

\dwrspace{1}

\question Find the general solution of the equation 
    \[
        ty'' - (t+1)y' + y = t^2
    \]
given that $y_1 = e^t$ and $y_2 = t+1$ are solutions of the associated homogeneous equation. \emph{Hint.}~Use the method of variation of parameters. 

\dwrspace{1}

\end{questions}

\end{document}