\documentclass{beamer}

\mode<presentation>
{
  \usetheme{Szeged}      % or try Darmstadt, Madrid, Warsaw, ...
  \usecolortheme{crane} % or try albatross, beaver, crane, ...
  \usefonttheme{structurebold}  % or try serif, structurebold, ...
  \setbeamertemplate{navigation symbols}{}
  \setbeamertemplate{caption}[numbered]
} 

\usepackage{paralist}
\usepackage{xparse}
\usepackage[english]{babel}
\usepackage[utf8x]{inputenc}

\NewDocumentCommand\C{}{\mathbf{C}}

\title[Oscillations]{Oscillatory solutions to second-order homogeneous linear equations}
\author{D.\ Rosoff}
\institute{College of Idaho}
\date{18 March 2013}

\begin{document}

\begin{frame}
  \titlepage
\end{frame}

\section{Warm-up}
\begin{frame}[t]\frametitle{Warm-up}
    
Recall the formula from last time:
\pause
\begin{block}{Euler's formula}
    $$e^{\lambda + i \mu} = e^{\lambda} (\cos{\mu} + i \sin{\mu}).$$
\end{block}
\pause
Respect. Use Euler's formula to write each of the following complex numbers in the form $a + ib$, called the \emph{rectangular form} of the number.

\begin{align*}
    \exp(1 + 2i) \quad \quad & \exp(2-3i) \\
    \exp(2-(\pi/2)i) \quad \quad & \exp{2\pi i}
\end{align*}


\end{frame}
\section{Recap}

\begin{frame}\frametitle{Last time}
  Last time we introduced enough complex algebra to solve the linear homogeneous differential equation

  \[
    ay'' + by' + cy = 0
  \]

  when $D = b^2 - 4ac < 0$.

  You used Euler's formula and the standard ``exponential trick''.
\end{frame}

\begin{frame}[t]\frametitle{Last time}
    
  Specifically, you found two \emph{complex-valued} solutions to $ay'' + by' + cy = 0$. If $r_1$ and $r_2$ are the two complex roots of the characteristic equation, we can write $r_j = \lambda \pm i \mu $. Then Euler's formula gives
  \begin{align*}
      y_1 &= e^{\lambda t} (\cos{(\mu t)} + i \sin{(\mu t)}) \\
      y_2 &= e^{\lambda t} (\cos{(-\mu t)} + i \sin{(-\mu t)}) \\
                      &= e^{\lambda t} (\cos{(\mu t)} - i \sin{(\mu t)}),
  \end{align*}
  where in the last step we have used the identities 
  \[
    \cos{(-x)} = \cos{(x)} \text{ and } \sin{(-x)} = -\sin{(x)}.
  \]
\end{frame}

\begin{frame}[t]\frametitle{Getting real solutions}
    Of course we are most interested in real-valued solutions.

    You found that
    \begin{align*}
        u &= \frac{y_1 + y_2}{2}, \\
        v &= \frac{y_1 - y_2}{2i}
    \end{align*}

    are real-valued (all the $i$-stuff cancels away). These functions are evidently linear combinations of solutions. So they are solutions to the differential equation: that is, we have

    \[
        au'' + bu' + cu = 0, \quad av'' + bv' + cv = 0.
    \]

\end{frame}
\section{Negative discriminants}

\begin{frame}[t]\frametitle{Fundamental solutions when $D < 0$}
    
    The method you stepped through in the worksheet is completely general, and yields a fundamental system of solutions (you will check this for yourself in the exercises).

    \begin{itemize}
        \item First, find the roots of the characteristic polynomial. They have the form $\lambda \pm i \mu$, for real numbers $\lambda$ and $\mu$.
        \item Write down $u = e^{\lambda t} \cos{(\mu t)}$ and $v = e^{\lambda t} \sin{(\mu t)}$. 
        \item The Wronskian of $u$ and $v$ is nonzero, so every solution to $ay'' + by' + cy = 0$ is a linear combination of $u$ and $v$.
        \item If necessary, use initial conditions to determine the appropriate coefficients on $u$ and $v$. Since the Wronskian is nonzero, there is exactly one choice of coefficients.
    \end{itemize}

\end{frame}

\begin{frame}[t]\frametitle{Exercises}
    
    Find the general solution of the differential equation.
    \begin{itemize}
        \item $y'' - 2y' + 6y = 0$
        \item $y'' + 2y' + 2y = 0$
        \item $4y'' + 9y = 0$
        \item $9y'' + 9y' - 4y = 0$
        \item $y'' + 4y' + (25/4)y = 0$
    \end{itemize}
    Solve the initial value problem.
    \begin{itemize}
        \item $y'' + 4y' + 5y = 0, \quad y(0) = 1, \; y'(0) = 0.$
        \item $y'' + y = 0, \quad y(\pi/3) = 2, \; y'(\pi/3) = -4.$
    \end{itemize}
\end{frame}
\end{document}