\def\encoding{UTF-8}
\input{mmd-beamer-header-rosoff}
\def\mytitle{Reduction of Order}
\def\subtitle{A generalization of the repeated-roots technique}
\def\affiliation{The College of Idaho}
\def\mydate{22 March 2013}
\def\latexmode{beamer}
\def\fonttheme{structurebold}
\def\colortheme{crane}
\def\theme{Szeged}
\input{mmd-beamer-begin-doc-rosoff}
\section{Recap}
\label{recap}

\begin{frame}

\frametitle{Last time: repeated roots}
\label{lasttime:repeatedroots}

Last time, we completed our investigation of the second-order linear homogeneous equation
\[ ay'' + by' + cy = 0 \]
by finding a general solution to the equation in the last remaining case, in which $ D = b^2 - 4ac = 0 $. 

A new technique was necessary, because the exponential trick only gave us half of the general solution, the 1-dimensional family $ cy_1 = ce^{(-b/2a)t} $.

\end{frame}

\begin{frame}

\frametitle{Last time: guess-and-check}
\label{lasttime:guess-and-check}

In the group worksheet, you found that if $ v(t) $ is an unknown function, then $ v y_1 $ is a solution to the DE---that is,
\[ a(v y_1)'' + b(v y_1)' + c(v y_1) = 0, \]
---if and only if $ v'' = 0 $. A bit of calculus and a moment's reflection shows that $ v = \alpha t + \beta $ in this case.

Thus you generated a new class of solutions, the functions
\[ (\alpha t + \beta) e^{(-b/2a)t}. \]

Notice! it's a 2-dimensional family. Maybe it's the general solution?

\end{frame}

\begin{frame}

\frametitle{Last time: the general solution}
\label{lasttime:thegeneralsolution}

\begin{itemize}
\item Now, it is true that this new class of solutions is a suitable ``other half'' of our general solution in the sense that its Wronskian with $ y_1 $ is everywhere nonzero. But it is not optimally efficient, because $ y_1 $ appears in the new class! Put $ \alpha = 0 $ and $ \beta = 1 $ to obtain it.

\item The part of your solution that is fundamentally ``new'' is $ y_2 = c_2 te^{(-b/2a)t} $. You can check that the Wronskian $ W(y_1, y_2) $ is still nonzero, so that these two functions generate all solutions to the differential equation through linear combinations.

\item It is equally true that the new solution $ \alpha y_2 + \beta y_1 $ ``is'' the general solution. There's no contradiction there.

\end{itemize}

\end{frame}

\section{Reduction of order for nonconstant coefficients}
\label{reductionoforderfornonconstantcoefficients}

\begin{frame}

\frametitle{The magic secret of Wronskians}
\label{themagicsecretofwronskians}

You might be a bit uncomfortable about my blithe replacement of
\[ (\alpha t + \beta) e^{(-b/2a)t}  = \alpha t e^{(-b/2a)t} + \beta e^{(-b/2a)t}\]
with
\[ c_2 t e^{(-b/2a)t}. \]

\begin{itemize}
\item The fact is, linear algebra provides us with a general theory of why we can make replacements like this, but we haven't the time to go into it.
 \pause 

\item The great thing is, we don't need to! \emph{Once we have a nonzero Wronskian, it makes no difference where the solutions came from.} We can compute them, \pause guess them, \pause or find them scrawled in blood on the walls of an ancient tomb. The Wronskian doesn't care, and tells us definitively that we have solved the equation.

\end{itemize}

\end{frame}

\begin{frame}

\frametitle{The general method}
\label{thegeneralmethod}

\begin{itemize}
\item There was a bit of confusion last time---I promised a \emph{first-order} equation for you to solve, and didn't deliver, since the promised equation for $ v $ turned out to look like a second-order one.

\item Whenever we have found---again, by any means necessary---a single solution to a second-order linear homogeneous ODE (even one with nonconstant coefficients), the method of ``promoting $ c $ to a function'' can get us the other half.

\end{itemize}

\end{frame}

\begin{frame}

\frametitle{The general method: in general}
\label{thegeneralmethod:ingeneral}

\begin{itemize}
\item Suppose that, in the tomb of an ancient king, we find the following equations scrawled in blood on the wall in a shaky but regal hand:
\[ 2t^2 y'' + 3t y' - y = 0, \quad y_1 = t^{-1}. \]
Evidently $ y_1 $ is a solution (easily checked), and since the ODE is linear and homogeneous, we know that $ cy_1 $ is also a solution for all real $ c $.

\item We will use the method of reduction of order---promoting $ c $ to a function---to obtain another solution $ y_2 $ that is the missing piece of the general solution; in other words, we will have
\[ W(y_1, y_2) \ne 0. \]

\end{itemize}

\end{frame}

\begin{frame}

\frametitle{Using the method}
\label{usingthemethod}

We write $ y_2 = vy_1 $ and substitute in. Observe that $ (vy_1)' = v'y_1 + vy'_1 $ and that $ (vy_1)'' = v''y_1 + 2v'y'_1 + vy''_1 $. Substitution back into the original differential equation then gives us
\begin{align*}
0 &= 2t^2 (v''y_1 + 2v'y'_1 + vy''_1) + 3t (v'y_1 + vy'_1) - vy_1 \\
  &= (2t^2 y_1) v'' + (4t^2 y'_1 + 3t y_1) v' + (2t^2 y''_1 + 3ty'_1 - y_1) v \\
  &= (2t^2 y_1) v'' + (4t^2 y'_1 + 3t y_1) v'.
\end{align*}

The coefficient of $ v $ is zero, because $ y_1 $ is a solution to the ODE!

\end{frame}

\begin{frame}

\frametitle{The method continues}
\label{themethodcontinues}

Now we can see how the order has been reduced. The equation
\[ (2t^2 y_1) v'' + (2t^2 y'_1 + 3t y_1) v' = 0 \]
is, admittedly, a second-order equation in $ v $. But it is a first-order equation in $ v' $, in fact a separable one. The textbook details the solution of this equation---much easier, if we put $ y_1 = t^{-1} $ throughout.

\begin{itemize}
\item I suggest you check for yourself, without looking in the text, if possible, that in this example we find $ v = t^{1/2} $ and therefore $ y_2 = t^{-1/2} $.

\end{itemize}

\end{frame}

\begin{frame}

\frametitle{SPRING BREAK}
\label{springbreak}

{\Huge \textsc{Have a great break!}}\dwrspace{1}

\begin{itemize}
\item {\ldots}and be ready for more guess-and-check fun when we come back on the 1st.

\item We'll start with section 3.5.

\item Some daily problems are assigned from the last few sections.

\end{itemize}

\end{frame}

\mode<all>
\input{mmd-beamer-footer}

\end{document}\mode*

