\def\encoding{UTF-8}
\input{mmd-beamer-header-rosoff}
\def\mytitle{Systems of equations}
\def\affiliation{The College of Idaho}
\def\myauthor{Math 352 Differential Equations}
\def\mydate{6 May 2013}
\def\latexmode{beamer}
\input{mmd-beamer-begin-doc-rosoff}
\def\htmlheaderlevel{2}
\section{Converting an $ n $th order equation}
\label{convertingannthorderequation}

\begin{frame}

\frametitle{Converting to a system}
\label{convertingtoasystem}

We can convert the second-order homogeneous linear equation
\[
    mu'' + \gamma u' + ku = 0
\]
into a $ 2 \times 2 $ system by letting $ x_1 = u $, $ x_2 = u' $. The equation above then becomes
\[
    mx'_2 + \gamma x_2 + kx_1 = 0,
\]
so a $ 2 \times 2$ system whose solutions are the same as the second-order equations' solutions is
 
    \begin{align*}
        x'_1 &= x_2 \\
        x'_2 &= -\frac{k}{m} x_1 - \frac{\gamma}{m} x_2
    \end{align*}


\end{frame}

\begin{frame}

\frametitle{Warm-up}
\label{warm-up}

Solve 1, 3, 5 from section 7.1. When you have a choice to make, always write the choice with \emph{fewer primes}. When you are transforming an initial value problem, don't forget to transform the initial conditions!

\end{frame}

\section{Procedure: $ 2 \times 2 $ homogeneous}
\label{procedure:2times2homogeneous}

\begin{frame}

\frametitle{Algorithm for solving $ 2 \times 2$ homogeneous}
\label{algorithmforsolving2times2homogeneous}

\begin{itemize}
\item Obtain coefficient matrix $ A $

\item Find eigenvalues: the roots of $ \det(A - \lambda I) $

\item Solve for the coefficient vectors: the eigenvectors of $ A $

\item Write down the solution functions.

\end{itemize}

\end{frame}

\begin{frame}

\frametitle{Example}
\label{example}

Consider the system of differential equations
 
\begin{align*}
    x'_1 &= x_1 + x_2 \\
    x'_2 &= 4x_1 + x_2
\end{align*}

In matrix form, this becomes
\[
    \vec{x}' = \begin{pmatrix}
        1 & 1 \\
        4 & 1
    \end{pmatrix} \vec{x}.
\]
Write $ A $ for the coefficient matrix.

\begin{itemize}
\item Check that $ \det{A - \lambda I} = \lambda^2 - 2 \lambda -3 $, so that the eigenvalues of $ A $ are $ \lambda_1 = 3 $, $ \lambda_2 = -1 $.

\end{itemize}

\end{frame}

\begin{frame}

\frametitle{Getting the eigenvectors}
\label{gettingtheeigenvectors}

\begin{itemize}
\item Solve the modified system for each eigenvalue to obtain eigenvectors $ \vec{\xi}^{(1)} $ and $ \vec{\xi}^{(2)} $.

\end{itemize}

The modified system is
\[
    \begin{pmatrix}
        1 - \lambda & 1 \\
        4 & 1 - \lambda
    \end{pmatrix} \begin{pmatrix} \xi_1 \\ \xi_2 \end{pmatrix} = \begin{pmatrix} 0 \\ 0 \end{pmatrix}.
\]

\begin{itemize}
\item When $ \lambda = 3 $, we see that the system reduces to the equation $ -2 \xi_1 + \xi_2 = 0 $, so an eigenvector for $ \lambda = 3 $ is
\[
\vec{\xi}^{(1)} = \begin{pmatrix} 1 \\ 2 \end{pmatrix}.
\]

\end{itemize}

\end{frame}

\begin{frame}

\frametitle{The other eigenvector}
\label{theothereigenvector}

When $ \lambda = -1 $, we see that the system becomes
\[
    \begin{pmatrix}
        2 & 1 \\
        4 & 2
    \end{pmatrix} \begin{pmatrix} \xi_1 \\ \xi_2 \end{pmatrix} = \begin{pmatrix} 0 \\ 0 \end{pmatrix}.
\]

\begin{itemize}
\item Therefore an eigenvector for $ \lambda = -1 $ is $ \vec{\xi}^{(2)} = (1, -2)$.

\end{itemize}

\end{frame}

\begin{frame}

\frametitle{Putting it all together}
\label{puttingitalltogether}

\begin{itemize}
\item The general solution to the system is, in the ``nice'' case, the general linear combination of the eigenvectors multiplied by the exponentials with growth constants given by their eigenvalues:

\end{itemize}

\[
    \vec{x} = c_1 \vec{x}^{(1)} + c_2 \vec{x}^{(2)} = c_1 \begin{pmatrix} 1 \\ 2 \end{pmatrix} e^{3t} + c_2 \begin{pmatrix} 1 \\ -2 \end{pmatrix} e^{-t}.
\]

\begin{itemize}
\item Finally, we can write the solutions in scalar form. The functions $ x_1  $ and $ x_2 $ are the entries of the vector $ \vec{x} $.

\end{itemize}

\[
    x_1 = c_1 e^{3t} + c_2 e^{-t}, \quad x_2 = 2c_1 e^{3t} - 2c_2 e^{-t}.
\]

\end{frame}

\begin{frame}

\frametitle{Nice matrices}
\label{nicematrices}

What makes a matrix ``nice'' is too involved for us to describe completely. If the matrix has as many eigenvalues as it does columns, then it is automatically ``nice''.

\begin{itemize}
\item If $ \det{A - \lambda I} $ factors into \emph{distinct} linear factors, then $ A $ is ``nice''.

\item In more technical terms $ A $ is \emph{diagonalizable} if and only if $ A $ is ``nice''.

\end{itemize}

\end{frame}

\begin{frame}

\frametitle{Algorithm, again}
\label{algorithmagain}

\begin{itemize}
\item Obtain coefficient matrix $ A $.

\item Find eigenvalues: the roots of $ \det(A - \lambda I) $.

\item Solve for the coefficient vectors: the eigenvectors of $ A $.

\item Write down the solution functions $ x_1 $ and $ x_2 $.

\end{itemize}

\[
    \begin{pmatrix} x_1 \\ x_2 \end{pmatrix} = c_1 \vec{\xi}^{(1)} e^{\lambda_1 t} + c_2 \vec{\xi}^{(2)} e^{\lambda_2 t}.
\]

\begin{itemize}
\item Solve 1--7 odd, part (a) only, in section 7.5.

\end{itemize}

\end{frame}

\mode<all>
\input{mmd-beamer-footer}

\end{document}\mode*

