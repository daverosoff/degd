\def\encoding{UTF-8}
\input{mmd-beamer-header-rosoff}
\def\mytitle{Echelon form with Sage}
\def\affiliation{The College of Idaho}
\def\myauthor{Math 352 Differential Equations}
\def\mydate{1 May 2013}
\def\latexmode{beamer}
\input{mmd-beamer-begin-doc-rosoff}
\def\htmlheaderlevel{2  \\
\# Echelon form and Sage}
\begin{frame}

\frametitle{Transient and steady-state decomposition}
\label{transientandsteady-statedecomposition}

Our oscillations are controlled by the differential equation
\[
    mu'' + \gamma u' + ku = F(t).
\]
Today we will look briefly at the case $ \gamma > 0 $. When we think about the general solution to this forced, damped equation, we regard it as usual in two pieces:
\[
    u_c + U
\]
where $ u_c $ is the general solution of the associated homogeneous equation (the same system, but unforced) and $ U $ is a particular solution.

\end{frame}

\begin{frame}

\frametitle{Why ``transient'' and ``steady-state''?}
\label{whytransientandsteady-state}

\begin{itemize}
\item Observe that $ u_c \to 0 $ as $ t \to \infty $. Therefore the effects of the initial conditions are ephemeral, which is why we call $ u_c $ the \emph{transient} part of the solution, and $ U $ the \emph{steady-state} solution. If one returns to the system after time has passed, only the $ U $-behavior of the system is evident.

\item The steady-state $ U $ is also often called the \emph{forced response} of the system to the forcing function $ F $. 

\item Observe how the decomposition of the response into $ u_c + U $ mirrors the decomposition of the system into internal ($ m $, $ \gamma $, $ k $, $ c_1 $, $ c_2 $) and external ($ F(t) $) factors. This is what mathematicians and physicists mean by \emph{elegance}.

\end{itemize}

\end{frame}

\begin{frame}

\frametitle{The solution}
\label{thesolution}

\begin{itemize}
\item As before, if we assume that $ F(t) = F_0 \cos{(\omega t)} $, we may write $ U(t) = R \cos{(\omega t - \delta)} $ for some amplitude $ R $ and phase shift $ \delta $. These constants are determined by the formulas
\[
R = \frac{F_0}{\Delta}, \quad \cos \delta = \frac{m(\omega_0^2 - \omega^2)}{\Delta}, \quad \sin \delta = \frac{\gamma \omega}{\Delta},
\]
where $ \Delta = \sqrt{m^2(\omega_0^2 - \omega^2)^2 + \gamma^2 \omega^2} $.

\item We can use these formulas to compare $ R $, the amplitude of the forced response, to $ F_0/k $, the length by which the spring is stretched when subjected to a constant force of magnitude $ F_0 $. This depends on $ \omega $.

\end{itemize}

\end{frame}

\begin{frame}

\frametitle{Amplitude from frequency}
\label{amplitudefromfrequency}

\begin{itemize}
\item It turns out that, if we let $ \Gamma = \gamma^2/mk $ (so that it is a constant multiple of $ Q $ from last time), we obtain
\[
\frac{Rk}{F_0} = \left( \left( 1 - \frac{\omega^2}{\omega_0^2} \right)^2 + \Gamma \frac{\omega^2}{\omega_0^2} \right)^{-1/2}.
\]

\item When $ \omega \approx 0 $, this number is very close to $ 1 $, or in other words, $ R \approx F_0/k $. This matches with our physical intuition.

\item When $ \omega \gg 0 $, $ R $ is small. Extremely high frequency excitation produces a negligible vibration (think of the child kicking his feet in the swings).

\end{itemize}

\end{frame}

\begin{frame}

\frametitle{Where is the maximum amplitude?}
\label{whereisthemaximumamplitude}

\[
    \frac{Rk}{F_0} = \left( \left( 1 - \frac{\omega^2}{\omega_0^2} \right)^2 + \Gamma \frac{\omega^2}{\omega_0^2} \right)^{-1/2}
\]

\begin{itemize}
\item The right-hand side can be regarded as a function of $ \omega $. It should attain a maximum (the greatest forced amplitude achievable) for some $ \omega_{\mathrm{max}} $. Some tedious algebra shows that
\[
\omega_{\mathrm{max}} = \omega_0^2 - \frac{\gamma^2}{2m^2} = \omega_0^2 \left( 1 - \frac{\gamma^2}{2mk} \right).
\]

\item Observe that $ \omega_{\mathrm{max}} < \omega_0 $ and that when $ \gamma $ is small, these values are very close.

\item The textbook has a more detailed discussion of the variance of $Rk/F_0$ with $\omega/\omega_0$. I encourage you to read it closely.

\end{itemize}

\end{frame}

\begin{frame}

\frametitle{Wrap-up}
\label{wrap-up}

In particular, the amplitude achieved when $ \omega = \omega_{\mathrm{max}} $ is given by
\[
    R_{\mathrm{max}} = \frac{F_0}{\gamma \omega_0 \sqrt{1-Q}} \approx \frac{F_0}{\gamma \omega_0} \left( 1 + \frac{Q}{2} \right).
\]
Evidently, the approximation fails if $ Q > 1 $. In fact, if $ Q > 1/2 $, then $ \omega_{\mathrm{max}} $ is a complex number, and $ R $ is a monotone decreasing function of $ \omega $.

\end{frame}

\section{Our last topic: Higher-order equations}
\label{ourlasttopic:higher-orderequations}

\begin{frame}

\frametitle{Introduction to higher-order equations}
\label{introductiontohigher-orderequations}

In practice, the linear differential equation
\[
    a_n y^{(n)} + \cdots + a_1 y' + a_0 y = g(t)
\]
is solved by converting it to a system of $ n $ first-order differential equations.

Recall that the equation $ y' = ay $ has general solution of exponential type:
\[
    y = ce^{at}.
\]

A system of such equations also has a general solution of exponential type, as we shall see.

\end{frame}

\begin{frame}

\frametitle{Matrices and vectors}
\label{matricesandvectors}

The solution of such systems is facilitated by a new kind of algebra. An $ m \times n $ matrix is an array of numbers (for us, real numbers, also eventually complex numbers inevitably arise) like this:

\[
    A = \begin{pmatrix}
        1 & 4 & 6 & -3 \\
        2 & -2 & 0 & 7
    \end{pmatrix}.
\]

Here, $ m = 2 $ and $ n = 4 $. We always list the row-index first. The same goes if we want to leave the entries (the numbers in the matrix) anonymous:

\[
    A = \begin{pmatrix}
        a_{11} & a_{12} & a_{13} & a_{14} \\
        a_{21} & a_{22} & a_{23} & a_{24}
        \end{pmatrix}.
\]


\end{frame}

\mode<all>
\input{mmd-beamer-footer}

\end{document}\mode*

