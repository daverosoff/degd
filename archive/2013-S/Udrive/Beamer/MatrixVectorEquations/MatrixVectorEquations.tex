\def\encoding{UTF-8}
\input{mmd-beamer-header-rosoff}
\def\mytitle{Matrix-vector equations}
\def\affiliation{The College of Idaho}
\def\myauthor{Math 352 Differential Equations}
\def\mydate{3 May 2013}
\def\latexmode{beamer}
\input{mmd-beamer-begin-doc-rosoff}
\def\htmlheaderlevel{2}
 \renewcommand{\vec}[1]{\mathbf{#1}} 

\section{Matrix-vector equations}
\label{matrix-vectorequations}

\begin{frame}

\frametitle{Recap}
\label{recap}

Recall that the equation $ y' = ay $ has general solution of exponential type:
\[
    y = ce^{at}.
\]

A system of such equations also has a general solution of exponential type, as we shall see.

\end{frame}

\begin{frame}

\frametitle{Matrices and vectors}
\label{matricesandvectors}

We've seen how the $ m \times n $ system of equations
 
    \begin{align*}
    x_1 + 4x_2 + 6x_3 &= -3\\
    2x_1 - 2x_2 \phantom{+ 0x_3} &= 7 
    \end{align*}

corresponds to the augmented matrix

\[
    A = \begin{pmatrix}
        1 & 4 & 6 & -3 \\
        2 & -2 & 0 & 7
    \end{pmatrix}.
\]

This is very convenient if we want to solve for the $ x_j $, but there is another formulation we must also understand.

\end{frame}

\begin{frame}

\frametitle{Matrix-vector form}
\label{matrix-vectorform}

\begin{itemize}
\item Instead of leaving the $ x_j $ out entirely, the \emph{matrix-vector form} of the system compresses them into a vector $ \vec{x} = (x_1, \ldots, x_n) $. Similarly, we view the constants $ b_i $ on the right-hand side as a vector $ \vec{b} = (b_1, \ldots, b_m) $. 

\item According to our convention that vectors are always \emph{column matrices}, $ \vec{x} $ is $ n \times 1 $ while $ \vec{b} $ is $ m \times 1 $.

\item This means that, writing $ A $ for the $ m \times n $ matrix of coefficients (\emph{not} the same as the augmented matrix we used previously) the product $ Ax $ is defined, and has the same shape as the constant vector $ \vec{b} $.

\end{itemize}

\end{frame}

\begin{frame}

\frametitle{Matrix-vector form: reloaded}
\label{matrix-vectorform:reloaded}

You can check, using the definition of matrix multiplication, that a list of solutions $ x_1, \ldots, x_n $ to the $ m \times n $ system of equations

    \[
    \begin{pmatrix}
        a_{11} & a_{12} & \cdots & a_{1n} & b_1 \\
        a_{21} & a_{22} & \cdots & a_{2n} & b_2 \\
        \vdots & \vdots & \ddots & \vdots & \vdots \\
        a_{m1} & a_{m2} & \cdots & a_{mn} & b_m 
    \end{pmatrix}
    \]

is the same thing as a solution to the matrix-vector equation

    \[
    \begin{pmatrix}
        a_{11} & a_{12} & \cdots & a_{1n} \\
        a_{21} & a_{22} & \cdots & a_{2n} \\
        \vdots & \vdots & \ddots & \vdots \\
        a_{m1} & a_{m2} & \cdots & a_{mn} 
    \end{pmatrix}
    \begin{pmatrix}
        x_1 \\ x_2 \\ \vdots \\ x_n
    \end{pmatrix} = \begin{pmatrix}
        b_1 \\ b_2 \\ \vdots \\ b_m
    \end{pmatrix}.
    \]


\end{frame}

\begin{frame}

\frametitle{Matrix-vector form: unlocked}
\label{matrix-vectorform:unlocked}

Typically, we would suppress the coefficients and just write the last huge mess as $ A\vec{x} = \vec{b} $ where $ A $, $ x $, and $ b $ are as above. Remember: \emph{solution} means what it always has (at least since week 4): something that, when you plug it in to the equation, makes it true.

\begin{itemize}
\item From the matrix-vector point of view, we aren't plugging in a whole list of $ x_j $, but a single vector $ \vec{x} $.

\item General theory of matrix algebra (which we haven't time to develop) tells us that for a fixed $ A $ and $ \vec{b} $, there are three possibilities: no solution, one solution, or infinitely many solutions.

\end{itemize}

\end{frame}

\begin{frame}

\frametitle{Other questions about matrix-vector equations}
\label{otherquestionsaboutmatrix-vectorequations}

Instead of thinking of $ A $ and $ \vec{b} $ as fixed and asking about the set of solution vectors $ \vec{x} $, we might ask

\begin{itemize}
\item For a fixed $ A $, which $ m \times 1 $ vectors $ \vec{b} $ occur as values of $ Ax $?

\end{itemize}

This is another question we could answer by row-reduction techniques. But that is beyond the scope of this course.

\end{frame}

\begin{frame}

\frametitle{Unpacking the matrix-vector form}
\label{unpackingthematrix-vectorform}

Let's write $ A_j $ for the $ j $th column of $ A $ regarded as a vector, so that
\[
    A_j = \begin{pmatrix} a_{1j} \\ a_{2j} \\ \vdots \\ a_{mj} \end{pmatrix}.
\]
Thus $ A_j $ is a $ m \times 1 $ matrix, just like the right-hand side $ \vec{b} $ in the equation $ A \vec{x} = \vec{b} $.

\begin{itemize}
\item You can check, using the definition of matrix multiplication, that
$ A \vec{x} = x_1 A_1 + x_2 A_2 + \cdots + x_n A_n$.

\item This representation is key in what follows.

\end{itemize}

\end{frame}

\section{Eigenvalues and systems of linear differential equations}
\label{eigenvaluesandsystemsoflineardifferentialequations}

\begin{frame}

\frametitle{Application to ODEs}
\label{applicationtoodes}

Let's pass to the application of matrix theory we are interested in: systems of first-order linear ODEs. The algebra is similar to what we have done so far, but the $ x_j $ now must be regarded as differentiable \emph{functions} of some usually unwritten variable $ t $. The right-hand side $ \vec{b} $ is replaced by the vector of derivatives of the $ x_j $.

\[
    \begin{pmatrix}
        x'_1 \\ x'_2 \\ \vdots \\ x'_n
    \end{pmatrix} = \begin{pmatrix}
        a_{11} & a_{12} & \cdots & a_{1n} \\
        a_{21} & a_{22} & \cdots & a_{2n} \\
        \vdots & \vdots & \ddots & \vdots \\
        a_{m1} & a_{m2} & \cdots & a_{mn} 
    \end{pmatrix}
    \begin{pmatrix}
        x_1 \\ x_2 \\ \vdots \\ x_n
    \end{pmatrix}
\]

\begin{itemize}
\item We usually write this as $ \vec{x'} = A \vec{x} $. Observe that $ \vec{x} $ is now a variable vector and $ \vec{x'} $ stands for the vector of the derivatives.

\end{itemize}

\end{frame}

\begin{frame}

\frametitle{Assumptions, I}
\label{assumptionsi}

\begin{itemize}
\item Like before, the coefficients must be constant for our methods to work. Assume that $ A $ has constant entries (they are not functions of $ t $).

\item We also assume that $ A $ is square, that is, that $ m = n $. According to universal mathematical custom we write $ n $ for this common value.

\end{itemize}

\end{frame}

\begin{frame}

\frametitle{Assumptions, II}
\label{assumptionsii}

\begin{itemize}
\item Just like we guessed that solutions of the original linear first-order differential equation $ y' = ay $ would be of exponential type, we are going to guess the form of the entries of $ \vec{x} $.

\item In fact, just like we assumed the solutions of $ y' = ay $ would be multiples of the exponential $ e^{at} $, we'll assume the solutions of $ x' = Ax $ are \emph{vector} multiples of the exponential $ e^{\lambda t} $ for suitable $ \lambda $.

\item The $ \lambda $ that ``work'' are called the \emph{eigenvalues} of the matrix $ A $.

\end{itemize}

\end{frame}

\begin{frame}

\frametitle{Eigenvalues}
\label{eigenvalues}

\begin{itemize}
\item To find the eigenvalues of $ A $, we solve the linear system of ordinary algebraic equations
\[
\begin{pmatrix}
    a_{11} - \lambda & a_{12} & \cdots & a_{1n} \\
    a_{21} & a_{22} - \lambda & \cdots & a_{2n} \\
    \vdots & \vdots & \ddots & \vdots \\
    a_{n1} & a_{n2} & \cdots & a_{nn} - \lambda
\end{pmatrix} \begin{pmatrix}
    \xi_1 \\ \xi_2 \\ \vdots \\ \xi_n
\end{pmatrix} = \begin{pmatrix}
    0 \\ 0 \\ \vdots \\ 0
\end{pmatrix}.
\]

\end{itemize}

\end{frame}

\mode<all>
\input{mmd-beamer-footer}

\end{document}\mode*

