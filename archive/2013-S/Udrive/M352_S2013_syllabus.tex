\documentclass[11pt]{amsart}
\usepackage{
    %graphicx,
    %wrapfig,
    fontspec,
    amsmath,
    amsthm,
    amssymb,
    latexsym,
    paralist,
    enumerate,
    epigraph,
    nicefrac,
    booktabs,
}
\usepackage{rosoff}
\usepackage{tabularx,dcolumn}
\usepackage{hyperref}
\setlength{\parskip}{1.0ex} \setlength{\parindent}{0pt}
%\title{{\Large Mathematics~352, Spring~2013 \\ February~11}}
\setlength{\headheight}{13pt} \setlength{\headsep}{8pt}
\pagestyle{fancy}
\linespread{1.01}
\lhead{Mathematics 352}
\chead{Spring 2013}
\rhead{February 11}
\lfoot{} \cfoot{} \rfoot{}
\begin{document}
\thispagestyle{empty}
\begin{center}
    {\Large Syllabus for \textbf{Differential Equations}, Spring 2013}
\end{center}
%\newcommand{\officehours}{TBA}
\newcommand{\officehours}{T~10--12; W~2--3; F~1--2}

\begin{figure}[ht]
\centering
    \begin{minipage}{0.75\linewidth}
        \begin{quote}
            \emph{I didn't discover curves; I only uncovered them.}
        \end{quote}
        \begin{flushright}
            {\small\emph{Mae West}}
        \end{flushright}
    \end{minipage}
\end{figure}

\begin{tabularx}{\linewidth}{X@{\quad}X}
    \textbf{Meetings:} 8:00--9:00, MWF, Boone~104 
            & \textbf{Office hours:} \officehours \\
    \textbf{Instructor:} Dr.\ Dave Rosoff 
            & \textbf{Email:} \nolinkurl{drosoff@collegeofidaho.edu} \\
    \textbf{Office:} Boone Hall~102C 
            & \textbf{Twitter:} \nolinkurl{@DaveRosoff}
\end{tabularx}

\textbf{Text:} The text is \emph{Elementary Differential Equations} by Boyce and DiPrima, ninth edition. This book and its cousins have been the standard for two generations, so the world is awash in earlier editions. I don't recommend you use an edition earlier than the eighth. If you have an earlier edition of the text than the ninth, that is fine: but ascertaining the differences in the texts, especially differently numbered or new homework problems, \emph{will be your sole responsibility}. The same caveat applies to any international version. Websites exist that catalog differences between versions.

\textbf{Course objectives:} The College of Idaho catalog description for MAT~352 is: ``A study of the solution and applications of ordinary differential equations including systems of equations using matrix algebra.'' See below for a more detailed description.

\textbf{Grading:} Scores are computed as a weighted average, with the following weights: homework~$0.05 = 5\%$, attendance/participations/quizzes~$0.25 = 25\%$, three in-class exams~$0.51 = 51\%$, and final exam~$0.19 = 19\%$. Observe that the weights sum to~$1 = 100\%$. It is not possible to earn a passing grade in the course without scoring at least $50\%$ on the final.

\textbf{Homework:} I will assign a lot more homework than I will collect. You might decide this means that if you don't do most of it, I will never know. This is true in at most a limited sense: when the exam comes, I will probably find out whether you have been completing the homework or not. My goal is as follows: every student who successfully completes all of the homework problems and \emph{understands all the solutions} should be able to earn an A in this course. All quizzes and exams are designed with this in mind. Therefore, I hope you will agree that it is in your very best interest to complete all of the assigned work, regardless of whether it is turned in for credit. The course is designed so that you will do best if you work at a modest but constant pace throughout the term. Cramming might work too, but not as well---and not as permanently, which is really the point.

%Even it you are not turning it in, it is important to pay attention to the style of your writing and your presentation. Good mathematical writing is essential for anyone who wishes to think clearly about mathematics---sloppy writing \emph{invariably} reflects underlying sloppy thinking. The process of making your ideas and reasoning \emph{clear, complete, and unambiguously correct} is the most powerful amplifier of mathematical power there is. Hence y
%Your solutions should be composed in brilliant English prose (e.g., accepted scientific usage, more or less correct grammar and spelling, and above all \emph{complete sentences}) sprinkled with equations here and there. Solutions in the popular ``pile-of-equations'' style are to be avoided and will not get much credit. You must explain what is happening as the action unfolds. You should also avoid falling into a ``two-column'' format that you may have learned in a high-school geometry class. It is stilted, artificial, and not easier to read than a pile of equations. Weave text and equations together for a gentle presentation that doesn't leave the reader guessing.

\textbf{Quizzes:} Quizzes will be given \emph{every day} (well, almost) to help you make sure you are staying on top of the material. Quiz problems will come directly, or nearly so, from the assigned daily homework problems. Quizzes can not be made up.

\textbf{Exam dates:} As specified below, you will take three in-class exams as well as the traditional comprehensive final exam. Let me know \emph{immediately} if you foresee a conflict. See below also for information regarding make-up exams.
\begin{itemize}
   \item Exam~1 (tentative): Monday, March~4
   \item Exam~2 (tentative): Wednesday, April~3
   \item Exam~3 (tentative): Monday, April~29
   \item Final Exam: Wednesday, May~15, 1:30--4:30
\end{itemize}
%
\textbf{Make-ups:} I will only consider make-up exams with a \emph{documented, compelling reason} and sufficient (two weeks is always enough) notice; otherwise, remaining exams will be reweighted.  Quizzes can not be made up.

\textbf{Laptops, phones, and other screens:} Please do not use class time to work on a laptop or smart phone. Unless there is an emergency situation, phones should not be out during lecture. If you have a laptop out during lecture, it will be assumed that the laptop is being used to take notes. As such, it is expected that any student using a laptop during class will email a copy of these notes to~\nolinkurl{drosoff@collegeofidaho.edu} immediately following class. 

\textbf{Academic integrity:} Students are expected to complete all graded work in accordance with the College Honor Code. Plagiarism, cheating, or borrowing without proper credit will not be tolerated.  Violations of academic honesty can result in loss of credit on an assignment, failure on an exam, or failure in the course. A referral may be made to the Vice President for Academic Affairs for all parties involved in academic dishonesty.

\textbf{Course overview:} If you are considering a life as any kind of engineer, physical scientist, or applied mathematician, life is differential equations. The laws of the universe are all differential equations: problems involving fluid flow, current in electric circuits, population dynamics, pricing of financial derivatives\footnote{Financial derivatives are unrelated to the derivatives of calculus.}, quantum mechanics, heat transfer, relativity, etc., are all formulated in this language. They are ubiquitous and of paramount importance. I assume this is reason enough for you to be interested.

Here is a bit of mathematical philosophy. Numbers and points are the same. Ordinary numbers are points on a line; points in other spaces are different kinds of numbers. If the space the points live in is a Euclidean space like $\R^2$ or $\R^3$, these numbers are often called \emph{vectors}. You have spent many years solving equations involving both constants and variables whose values are ordinary numbers, and maybe a few weeks solving equations involving vectors. Usually, we refer to this kind of work as \emph{algebra}. 

Points, and therefore numbers, are zero-dimensional\footnote{Think of the head of the vector, not the whole arrow.}. They are pure locations. They do not have extent of any kind. A one-dimensional number is called a \emph{curve}. You can also think of a curve as a family of points, as we often do in calculus. These points are related---as are the members of any family. Often, they are related by an equation defining the curve. Many algebraic problems involve finding a special point on a curve: a special member of a family.

Differential equations also involve both constants and variables and need to be solved. But here, the constants, variables, and solutions are all one-dimensional: they are curves! called the integral curves of the differential equation or equations. The solution of differential equations is the understanding of the algebra of curves. Usually, the plane will be partitioned into a bunch of related curves that fit together in a family---related, just as in algebra, by an equation (now, a differential equation) defining the family. Many problems in differential equations involve finding a special curve in a family. These are called \emph{initial value problems}.
As in algebra, most equations cannot be solved. Equations that admit exact solutions are very special. We will investigate some of the elementary equations and classify them according to their form. We will also discuss systems of linear differential equations, which are solved via linear-algebraic techniques involving matrices.

\textbf{A note on studying math:} This class is all about practice. Learning to recognize the different types of equations and quickly and accurately recall the procedures needed to solve them is a matter of rote technique. If you have practiced enough, you will remember how to do it\footnote{This is the definition of ``enough''.}. If not, something else may happen. Of course, there are deep ideas lurking beneath the computational techniques. Awareness and understanding of these ideas also comes with practice.

\textbf{Disability statement:} Students who have documented disabilities as addressed by the Americans With Disabilities Act and who need any test or course materials to be furnished in an alternative format should notify me immediately (during the first week of class).  Reasonable efforts will be made to accommodate the needs of such students.
\end{document}
