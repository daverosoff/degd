\documentclass[answers,12pt]{exam}
% \usepackage{pslatex}
\usepackage[utf8]{inputenc}
\usepackage[T1]{fontenc}
\usepackage{xparse}
%\usepackage{graphicx}
%\DeclareGraphicsExtensions{.jpg, .png}
\usepackage{amsmath}
%\usepackage{amsfonts}
\usepackage{fourier,MnSymbol}
\usepackage{enumerate}
\usepackage{siunitx}
\usepackage{tikz}
\usetikzlibrary{arrows}
\NewDocumentCommand\N{}{\mathbf{N}}
\NewDocumentCommand\R{}{\mathbf{R}}
\NewDocumentCommand\Z{}{\mathbf{Z}}
\NewDocumentCommand\Q{}{\mathbf{Q}}
\NewDocumentCommand\C{}{\mathbf{C}}
\NewDocumentCommand\Arg{m}{\mathrm{Arg}\;#1}
\RenewDocumentCommand\Re{m}{\mathrm{Re}\;#1}
\RenewDocumentCommand\Im{m}{\mathrm{Im}\;#1}
\NewDocumentCommand\dwrspace{m}{\vspace*{\stretch{#1}}}

%%%%%%%%%%%%%%%%%%%%%%%%%%%%%%%%%%%%%%%%%%%%%%%%%%%%%%%%%
% Fill in these!!

\NewDocumentCommand\termID{}{Spring 2013} % FILL IN TERM ID eg `Spring 2012'
\NewDocumentCommand\courseID{}{Mathematics 352} % FILL IN COURSE ID eg `Mathematics 111'
\NewDocumentCommand\assignmentID{}{Exam 3} % FILL IN ASSIGNMENT ID eg `Quiz 3'
\NewDocumentCommand\dateID{}{April 29, 2013} % FILL IN DATE

%%%%%%%%%%%%%%%%%%%%%%%%%%%%%%%%%%%%%%%%%%%%%%%%%%%%%%%%%

\firstpageheader{}{}{}
\runningheader{\textbf{\termID}}
 {}
 {\textbf{\courseID}}
 %{\emph{Page \thepage~of \numpages}}
\runningheadrule
\setlength{\parskip}{1ex}
\setlength{\parindent}{0pt}
\pagestyle{head}

\begin{document}
\noindent
\textbf{{\large \courseID \\ \assignmentID}}

\noindent
\dateID; 60 minutes  \hfill Name: \underline{\hspace{3in}} 
% \hfill Name: \underline{\hspace{0.5in}Answers\hspace{2in}}

% \printanswers
\addpoints

\noindent
This exam is closed book; you can use a calculator (\emph{not} a mobile phone) but no other electronic aids or printed references. \emph{If the wording or intent of any question is unclear, please ask me to clarify.} I am not trying to confuse you with the problem statements.

You can use your own paper or the provided blank copy paper. \textsc{Do not write anything you want graded on the exam paper.} Please write your name on each page you hand in. Show all your reasoning and all pertinent calculations. \emph{Give all answers in exact form. Decimal approximations of any accuracy will not receive full credit.}

When you have finished the exam, place this cover sheet on top of it and fold the packet in half \emph{the long way, with your name facing out}.

\dwrspace{1}

\begin{center}
\textsc{You are strongly encouraged to read all the problems before beginning.}
\end{center}

\dwrspace{1}

\begin{figure}[h]
\centering
    \begin{tikzpicture}
        \path[draw] (0, 0) rectangle (2.6, 4);
        \path[draw] (0, 4) -- (2.0, 4.7) -- (2.0, 4);
    \end{tikzpicture}
\end{figure}
\dwrspace{1}

\begin{center}
    \gradetable
\end{center}

\dwrspace{1}

\begin{center}
    {\Large \emph{Good luck!}}
\end{center}

\newpage

\begin{center}
    {\Large \textsc{No Work On This Page}}

    {\Large \textsc{All Work on Separate Pages}}
\end{center}

\begin{questions}  

%%%%%%%%%% From SP 10? 11?
\question[20] Consider a mass-spring system. Suppose it is perturbed and released in such a way that the displacement function for the mass is the solution of the initial value problem
\begin{displaymath}
    2u'' + 0.6u' + 0.45 u = 0, \quad u(0) = 3, \quad u'(0) = 0.
\end{displaymath}
Solve the initial value problem to find $u(t)$.
\begin{solution}
    The characteristic equation is $2t^2 + 0.6t + 0.45 = 0$. Multiplying by $20$ yields $40t^2 + 12t + 9 = 0$. The roots are therefore
    \[
        r_1 , r_2 = -\frac{3}{20} \pm \frac{9}{20} i.
    \]
    The solution we want is therefore of the form
    \[
        u = c_1 e^{-3t/20} \cos{\left(\frac{9}{20} t\right)} + c_2 e^{-3t/20} \sin{\left(\frac{9}{20} t\right)}.
    \]
    Applying the first initial condition yields $c_1 = 3$. Making this change and computing $u'$, we find
    \[
        u' = \left(\frac{9}{20}c_1 - \frac{9}{20} e^{-3t/20} \right) \cos{\left(\frac{9}{20} t\right)} + \text{multiple of $\sin{\left(\frac{9}{20} t\right)}$},
    \]
    so that $c_1 = 1$. Putting it all together, the desired solution is
    \[
        \left(\sin\left(\frac{9}{20} \, t\right) + 3 \, \cos\left(\frac{9}{20} \, t\right)\right) e^{\left(-\frac{3}{20} \, t\right)}.
    \]
\end{solution}

\question[20]
\emph{In this problem, assume that the acceleration $g$ due to gravity is exactly $\SI{10}{m/s^2}$.}

\begin{parts}
    \part Consider a spring-mass system with a mass of $\SI{6}{\kilo\gram}$ that stretches the spring a length of $\SI{5/3}{\meter}$ when first attached. Suppose this system operates in a viscous medium that provides $\SI{60}{\newton}$ of resistive force to a mass moving at $\SI{2}{m/s}$. 

    Find the position function for the motion that results if the mass is moved $\SI{2}{\meter}$ below its equilibrium position and then released. Show all work. Classify the motion as overdamped, underdamped, or critically damped.
    \begin{solution}
        We are given $m = 6$. To find $\gamma$, we recall that \emph{viscous} means that the damping force is linearly dependent on the speed (and $\gamma$ is the proportionality constant). Thus $60 = 2\gamma$, so that $\gamma = 30$. Finally, we obtain $k$ from the relation $mg = kL$, where $L = 5/3$. Evidently $k = mg/L = 36$. Thus, the equation of motion is
        \[
            6u'' + 30 u' + 36 u = 0,
        \]
        with initial conditions $u(0) = 2$, $u'(0) = 0$.

        The characteristic equation is $6t^2 + 30t + 36 = 0$. Dividing by 6, we obtain $t^2 + 5t + 6 = 0$. The roots are evidently $-2$ and $-3$, so the displacement function is of the form
        \[
            u = c_1 e^{-2t} + c_2 e^{-3t}.
        \]

        Applying the first initial condition, we see that $c_1 + c_2 = 2$. Therefore
        \[
            u' = -2c_1 e^{-2t} - 3c_2 e^{-3t},
        \]
        and applying the second initial condition, we see that $-2c_1 - 3c_2 = 0$. Solving, one obtains $c_1 = -4$, $c_2 = 6$. The displacement is then
        $-4 \, e^{\left(-3 \, t\right)} + 6 \, e^{\left(-2 \, t\right)}$.
    \end{solution}
    \part How should the damping constant $\gamma$ be changed to obtain a critically damped system, if the mass and spring constant are left unchanged? (Give an explicit value for $\gamma$.)
    \begin{solution}
        We require that $\gamma^2 - 4km = 0$, where $k = 36$, $m = 6$. It appears that we must have $\gamma^2 = 4 \cdot 36 \cdot 6 = 864$, or $\gamma = \sqrt{864} = 12 \sqrt{6}$.
    \end{solution}
\end{parts}
\question Use whatever method you like to find the general solution to the equation
\[
    y'' - 10y' + 25y = e^{5t}.
\]
\begin{solution}
    The characteristic polynomial is $(r - 5)^2$, so the general yoga of tells us that the complementary solution is $y_c = c_1 e^{5t} + c_2 t e^{5t}$. Undetermined coefficients then informs us that a particular solution to the equation will be of the form $Y = At^2 e^{5t}$. We compute $Y' = (5At^2 + 2At) e^{5t}$ and $Y'' = (25At^2 + 20At + 2A)e^{5t}$. Substituting, we obtain
    \[
        Y'' - 10Y' + 25Y = 2A e^{5t}.
    \]
    Therefore $A = 1/2$. The general solution is the sum
    \[
        y_c + Y = c_1 e^{5t} + c_2 t e^{5t} + \frac{1}{2} t^2 e^{5t}.
    \]

    If we foolishly insist on using variation of parameters, then the functions $y_1$ and $y_2$ are the fundamental set of solutions from above, $y_1 = e^{5t}$, and $y_2 = te^{5t}$. The Wronskian is
    \[
        W(y_1, y_2) = y_1 y'_2 - y'_1 y_2 = e^{10 t}.
    \]
    Therefore, the general solution is $u_1 y_1 + u_2 y_2$, where
    \[
        u_1 = \int -\frac{y_2 g}{W} \; dt, \quad u_2 = \int \frac{y_1 g}{W} \; dt.
    \]
    One finds that the integrands are $-t$ and $1$ respectively, so that
    \[
        u_1 = \int -t \; dt = -\frac{1}{2} t^2, \quad u_2 = \int 1 \; dt = t.
    \]
    The general solution is 
    \[
        y = -\frac{t^2}{2} e^{5t} + t(te^{5t}) = \frac{1}{2} t^2 e^{5t},
    \]
    as before.
\end{solution}
\question Find the general solution of the equation 
    \[
        ty'' - (t+1)y' + y = t^2
    \]
given that $y_1 = e^t$ and $y_2 = t+1$ are solutions of the associated homogeneous equation. \emph{Hint.}~Use the method of variation of parameters.
\begin{solution} %%%%%%%%%%%%%%%%%%%%%%% THIS PROBLEM WAS COMPLETELY FUCKED
    Let $W = y_1 y'_2 - y'_1 y_2$ as usual. Then $W = e^t - (t+1) e^t = -te^t$. The variations of parameters formulas tell us that a particular solution of the equation will be $u_1 y_1 + u_2 y_2$, where
    \[
        u_1 = \int \frac{-(t+1) t^2}{te^t} \; dt, \quad u_2 = \int \frac{e^t t^2}{te^t} \; dt.
    \]
    The first integral is annoying. Using integration by parts twice, we find that 
    \[
        u_1 = -\frac{t^2 + 3t + 3}{e^{t}}.
    \]
    For the second, we blessedly obtain 
    \[
        u_2 = -\frac{1}{2} t^2
    \]
    almost immediately. Therefore a particular solution is
    \[
        u_1 y_1 + u_2 y_2 = -(t^2 + 3t + 3) - \frac{1}{2} (t^3 + t^2),
    \]
    and the general solution is
    \[
        c_1 e^t + c_2 (t+1) - t^2 + 3t + 3 - \frac{1}{2} (t^3 + t^2).
    \]
\end{solution}
\end{questions}

\end{document}