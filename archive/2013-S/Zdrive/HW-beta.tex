\def\encoding{UTF-8}
\input{mmd-article-header}
\def\mytitle{Math 352 Homework and Quiz Schedule}
\def\myauthor{Dave Rosoff}
\def\mydate{February 6, 2013}
\def\htmlheaderlevel{1}
\def\latexmode{article}
\input{mmd-article-begin-doc}
\chapter{Math 352 Differential Equations  Homework, reading, and quizzes}
\label{math352differentialequationsbrhomeworkreadingandquizzes}

\section{Daily homework and reading}
\label{daily}

Looking for the weekly problem sets (\autoref{weekly}) or the quizzes (\autoref{quizzes})? Or the \href{sage.html}{Sage tutorial}\footnote{\href{sage.html}{sage.html}}? You can also \href{HW.pdf}{download an experimental PDF}\footnote{\href{HW.pdf}{HW.pdf}} of this homework page.

\subsection{Daily Problems, Week 6: March 18--March 22}
\label{dailyproblemsweek6:march18--march22}

\begin{itemize}
\item Assigned Wednesday, March 20. Read section 3.4 carefully, paying special attention to page 168 and to the last subsection \emph{Reduction of Order}.

\begin{itemize}
\item Section 3.4: Exercises 1--15, odd.

\end{itemize}

\item Assigned Monday, March 18. Skim section 3.4. Quiz Wednesday, March 20 over 3.2., 3.3. You can also view the (short, mostly recap) \href{OscillatorySolutions.pdf}{presentation}\footnote{\href{OscillatorySolutions.pdf}{OscillatorySolutions.pdf}}.

\begin{itemize}
\item Section 3.3: Exercises 8--22, even.

\end{itemize}

\end{itemize}

\subsection{Daily Problems, Week 5: March 11--March 15}
\label{dailyproblemsweek5:march11--march15}

\begin{itemize}
\item Assigned Friday, March 15. Read section 3.3.

\begin{itemize}
\item Section 3.2: Exercises 24--27.

\item Do all of the Exercises in the \href{GWComplexIntro_M352_S13.pdf}{Introduction to Complex Numbers}\footnote{\href{GWComplexIntro_M352_S13.pdf}{GWComplexIntro\_M352\_S13.pdf}} group assignment. You may find the \href{IntroductionComplexNumbers.pdf}{presentation from today's class}\footnote{\href{IntroductionComplexNumbers.pdf}{IntroductionComplexNumbers.pdf}} helpful as well.

\end{itemize}

\item Assigned Wednesday, March 13. Here is Wednesday's presentation about \href{WronskiansGeneralSolutions_M352_S13.pdf}{Wronskians and general solutions to second-order linear homogeneous equations}\footnote{\href{WronskiansGeneralSolutions_M352_S13.pdf}{WronskiansGeneralSolutions\_M352\_S13.pdf}}.

\begin{itemize}
\item Section 3.2: Exercises 1--6, 13, 14.

\end{itemize}

\item Assigned Monday, March 11. Quiz Friday, March 15 over 2.7 and 3.1. Here is the \href{GWLinearSecondOrder_M352_S13.pdf}{group worksheet}\footnote{\href{GWLinearSecondOrder_M352_S13.pdf}{GWLinearSecondOrder\_M352\_S13.pdf}} we did about exponential solutions to a second-order linear homogeneous differential equation with constant coefficients and the superposition principle (Theorem 3.2.2 in the text.).

\begin{itemize}
\item Section 3.1: Exercises 1--15 odd.

\end{itemize}

\end{itemize}

\subsection{Daily Problems, Week 4: March 4--March 8}
\label{dailyproblemsweek4:march4--march8}

\begin{itemize}
\item Assigned Friday, March 8. You can download the \href{EulersMethod.sws}{Sage worksheet}\footnote{\href{EulersMethod.sws}{EulersMethod.sws}} showing how to implement Euler's Method as a loop. You'll need to upload the .sws file to the Sage server to open it in your account.

\begin{itemize}
\item Section 2.5: Exercises 3--5, 7, 10.

\item Section 2.6: Exercises 1--3, 7.

\item Section 2.7: Exercises 1a, 3a.

\end{itemize}

\item March 6: Section 2.6. No problems today.

\item March 4: Exam 1.

\end{itemize}

\subsection{Daily Problems, Week 3: February 25--March 1}
\label{dailyproblemsweek3:february25--march1}

\begin{itemize}
\item No daily problems for March 1. Exam 1 is coming up! It covers sections 2.1 through 2.4.

\item Assigned Wednesday, February 27. Read section 2.5.

\begin{itemize}
\item Complete the \href{GWModeling_M352_S13.pdf}{activity problems from in-class}\footnote{\href{GWModeling_M352_S13.pdf}{GWModeling\_M352\_S13.pdf}} to be handed in Friday in lieu of Quiz 4. No new daily problems from the text.

\end{itemize}

\item Assigned Monday, February 25. Read section 2.4 and skim 2.5.

\begin{itemize}
\item Section 2.4: Exercises 1--6, 7, 8. No quiz Wednesday.

\end{itemize}

\end{itemize}

\subsection{Daily Problems, Week 2: February 18--22}
\label{dailyproblemsweek2:february18--22}

\begin{itemize}
\item Assigned Friday, February 22. Read section 2.3 and skim section 2.4. Work exercises 2.3.1--2.3.10.

\item Assigned Wednesday, February 20. Exercises from class: 2.2.12, 22, 24, 28. Daily homework is to use Sage or another computer algebra system to plot the solutions to the IVPs. Use the plots to answer the questions about max\slash min and domains of solutions.

\item Assigned Monday, February 18. Read section 2.2 and skim section 2.3. You can \href{SeparationOfVariables.sws}{download}\footnote{\href{SeparationOfVariables.sws}{SeparationOfVariables.sws}} Monday's in-class Sage worksheet; ignore everything after the command consisting only of a zero. You'll need to save the .sws file to disk. Then, you can upload into your own Sage notebook. See the \href{sage.html}{Getting Started worksheet}\footnote{\href{sage.html}{sage.html}} for more.

\begin{itemize}
\item Section 2.2: Exercises 1--8 (all exercises) and 9--15 odd.

\end{itemize}

\end{itemize}

\subsection{Daily Problems, Week 1: February 11--15}
\label{dailyproblemsweek1:february11--15}

\begin{itemize}
\item Assigned Friday, February 15. Read section 2.1 and skim section 2.2.

\begin{itemize}
\item Section 2.1: Exercises 1--5 and 16. To produce the plots, I suggest you make use of \href{sage.html}{Sage}\footnote{\href{sage.html}{sage.html}}. In solving the differential equations themselves, you'll need to make heavy use of integration by parts.

\end{itemize}

\item Assigned Wednesday, February 13. Read section 2.1.

\begin{itemize}
\item Section 1.3: Exercises 7--9, 11, 12, 14. Work done in \href{sage.html}{Sage}\footnote{\href{sage.html}{sage.html}} is fine (except 14), but if you feel you are a little rusty with integrals, you could probably use the practice of doing these by hand. If some of the functions involved look unfamiliar, Google them.

\end{itemize}

\item Assigned Monday, February 11. Read Chapter 1. You're invited to download \href{SlopeFields.sws}{this Sage worksheet}\footnote{\href{SlopeFields.sws}{SlopeFields.sws}} and open it at the public \href{http://www.sagenb.org/}{Sage notebook server}\footnote{\href{http://www.sagenb.org/}{http:/\slash www.sagenb.org\slash }} (registration required) to experiment with slope fields. See: \href{sage.html}{Getting Started with Sage}\footnote{\href{sage.html}{sage.html}}.

\begin{itemize}
\item Section 1.1: Exercises 15--20, 21, 22.

\item Section 1.2: Exercises 5--8, 12--14.

\end{itemize}

\end{itemize}

\section{Weekly homework}
\label{weekly}

Looking for the daily homework (\autoref{daily})?

\subsection{Weekly 5: March 11--March 15}
\label{weekly5:march11--march15}

Due \sout{Monday, March 18}Wednesday, March 20.

\begin{itemize}
\item Section 2.7: 15, 19 (use of Sage is encouraged).

\item Section 3.1: 20, 24.

\end{itemize}

\subsection{Weekly 4: March 4--March 8}
\label{weekly4:march4--march8}

Due Wednesday, March 13.

\begin{itemize}
\item Section 2.5: 15. I encourage you to make use of Sage, although this problem should not be as challenging as some of the 2.3 problems to do by hand. It will help a lot if you think about part (a) in verbal context. What is the meaning of $K$? What is the meaning of $\tau$? How do $\alpha$, $\beta$, etc., correspond to numbers in part (a)? Put some effort into understanding the problem before you begin to solve it.

\item Section 2.6: 12, 16.

\end{itemize}

\subsection{Weekly 3: February 25--March 1}
\label{weekly3:february25--march1}

Due Monday, March 4.

\begin{itemize}
\item Section 2.4: exercises 22, 23, 25, 26.

\end{itemize}

\subsection{Weekly 2: February 18--February 22}
\label{weekly2:february18--february22}

Due Monday, February 25.

\begin{itemize}
\item Section 2.2: exercises 10--16 (even; use Sage or equivalent for plots)

\item Section 2.3: exercises 13, 18, 19.

\end{itemize}

\subsection{Weekly 1: February 11--February 15}
\label{weekly1:february11--february15}

Due Monday, February 18.

\begin{itemize}
\item Section 2.1: exercises 13, 15, 17, 19.

\end{itemize}

\section{Quiz dates}
\label{quizzes}

\begin{itemize}
\item Quiz 7: Wednesday, March 20 (sections 3.2, 3.3).

\item Quiz 6: Friday, March 15 (sections 2.7, 3.1).

\item Quiz 5: Wednesday, March 13 (sections 2.5, 6).

\item Quiz 4: Monday, February 25 (section 2.3).

\item Quiz 3: Wednesday, February 20 (section 2.2). This and all subsequent quizzes will be open-note, but no books or calculators.

\item Quiz 2: Friday, February 15.

\item Quiz 1: Wednesday, February 13.

\end{itemize}

\input{mmd-memoir-footer}

\end{document}
