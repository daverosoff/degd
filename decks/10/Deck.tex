\documentclass[11pt,ignorenonframetext,]{beamer}
\usetheme{Szeged}
\usecolortheme{wolverine}
\usefonttheme{structurebold}
\usepackage{amssymb,amsmath}
\usepackage{ifxetex,ifluatex}
\usepackage{fixltx2e} % provides \textsubscript
\ifxetex
  \usepackage{fontspec,xltxtra,xunicode}
  \defaultfontfeatures{Mapping=tex-text,Scale=MatchLowercase}
\else
  \ifluatex
    \usepackage{fontspec}
    \defaultfontfeatures{Mapping=tex-text,Scale=MatchLowercase}
  \else
    \usepackage[utf8]{inputenc}
  \fi
\fi
\usepackage{listings}

% Comment these out if you don't want a slide with just the
% part/section/subsection/subsubsection title:
% \AtBeginPart{
%   \let\insertpartnumber\relax
%   \let\partname\relax
%   \frame{\partpage}
% }
% \AtBeginSection{
%   \let\insertsectionnumber\relax
%   \let\sectionname\relax
%   \frame{\sectionpage}
% }
% \AtBeginSubsection{
%   \let\insertsubsectionnumber\relax
%   \let\subsectionname\relax
%   \frame{\subsectionpage}
% }

\setlength{\parindent}{0pt}
\setlength{\parskip}{6pt plus 2pt minus 1pt}
\setlength{\emergencystretch}{3em}  % prevent overfull lines
\setcounter{secnumdepth}{0}

%%% begin dwr insert
\usepackage{patchcmd}
\usepackage{tabulary}   % Support longer table cells
\usepackage{booktabs}   % Support better tables
\usepackage[sort&compress]{natbib}

\usepackage{framed}     % Allow background color for images
\definecolor{shadecolor}{named}{white}

%\usepackage{paralist}
\usepackage{xparse}
\usepackage{subfigure}
\usepackage{hyperref}
%%% end dwr insert
\usepackage[beamer]{rosoff}
\title{A general method: variation of parameters}
\author{Math 352 Differential Equations}
\date{April 2, 2014}


\begin{document}
\frame{\titlepage}

\section{Introduction}

\begin{frame}\frametitle{Introduction: This one goes out to all the
$g$s}

The method of undetermined coefficients is very useful when the
derivatives of the terms appearing in $g(t)$ have a regular, predictable
shape and are themselves the derivatives of similar functions. But it's
not too hard to find functions $g(t)$ that would demand excessive
ingenuity of us when guessing the form of $Y$.

\end{frame}

\begin{frame}[fragile]\frametitle{Introduction: Too many functions!}

Consider $y'' + y' + y = 3 \csc{(t)}$. You can check that a particular
solution $Y$ is given by the formula\only\textless{}2\textgreater{}\{

\begin{equation*}
        -3 \sin{(t)} \cos{(2t)} + \frac{3}{2} \ln |\csc{(t)} - \cot{(t)}| ]
        \sin{(2t)} + 3 \cos{(t)} \sin{(2t)}.
    \end{equation*}

It's hard to imagine rules for undetermined coefficients that are both

\begin{verbatim}
- comprehensive enough to include a function like this and
- possible to commit to memory.
\end{verbatim}

\}

\end{frame}

\section{Variation of parameters}

\begin{frame}\frametitle{Variation of parameters: not a new thought}

Recall how we discovered the second fundamental solution $t e^{rt}$ in
the case of a repeated root $r$ of the characteristic equation. We
promoted a constant to a functional coefficient and looked for a
solution of the form $v(t) \exp{(rt)}$.

\begin{itemize}
\itemsep1pt\parskip0pt\parsep0pt
\item
  In other words, we allowed a parameter to vary.
\end{itemize}

We can try a similar idea when tackling the inhomogeneous equation
$y'' + q(t)y' + r(t)y = g(t)$, provided we have already solved the
associated homogeneous equation $y'' + q(t)y' + r(t)y = 0$. Let us write
$y_1$ and $y_2$ for a fundamental set of solutions as usual.

\end{frame}

\begin{frame}\frametitle{Variation of parameters: execution}

We shall search for a particular solution $Y$ of our inhomogeneous
equation of the form $Y = u_1 y_1 + u_2 y_2$: that is to say,

\begin{equation*}
    Y(t) = u_1(t) y_1(t) + u_2(t) y_2(t),
\end{equation*}

where $u_1$ and $u_2$ are unknown coefficient functions we will have to
find. We'll be plugging this expression for $Y$ along with its
derivatives back into the inhomogeneous equation, so let's compute the
derivatives now.

\end{frame}

\begin{frame}\frametitle{An unjustified assumption}

For no very good reason, let's assume that the fundamental solutions
$y_1$ and $y_2$ and the derivatives $u'_1$ and $u'_2$ satisfy the
following equation:

\begin{equation*}
    u'_1 y_1 + u'_2 y_2  = 0.
\end{equation*}

\begin{itemize}
\itemsep1pt\parskip0pt\parsep0pt
\item
  This is the worst rabbit-out-of-the-hat of the term.
\item
  Sorry.
\item
  The idea is evidently due to Lagrange\footnote{Joseph-Louis Lagrange
    (25 January 1736--10 April 1813) is most remembered for
    contributions analysis, number theory, and classical and celestial
    mechanics. His \emph{M'ecanique Analytique} was fundamental for the
    physicists of the 19th century.}, and it greatly simplifies the
  computation to come.
\item
  At no point does it cause problems of any kind.
\end{itemize}

\end{frame}

\begin{frame}\frametitle{The derivatives of $Y$}

Since

\begin{equation*}
    Y(t) = u_1(t) y_1(t) + u_2(t) y_2(t),
\end{equation*}

we also have

\begin{align*}
    Y'  &= u'_1 y_1 + u_1 y'_1 + u'_2 y_2 + u_2 y'_2, \\
        &= u_1 y'_1 + u_2 y'_2, \quad \text{and so} \\
    Y'' &= u_1 y''_1 + u'_1 y'_1 + u'_2 y'_2 + u_2 y''_2.
\end{align*}

\end{frame}

\begin{frame}\frametitle{Substitute back in}

Now we find that $Y''(t) + q(t)Y'(t) + r(t)Y(t)$ reduces to

\begin{equation*}
    (u_1 y''_1 + u'_1 y'_1 + u'_2 y'_2 + u_2 y''_2) + q(t)(u_1 y'_1 + u_2 y'_2)
    + r(t)(u_1 y_1 + u_2 y_2).
\end{equation*}

Collecting terms along $u_1$, $u_2$, and their derivatives, we get

\begin{align*}
    (y''_1  & + qy'_1 + ry_1)u_1 + (y''_2 + qy'_2 + ry_2)u_2 + y'_1 u'_1 + y'_2 u'_2 \\
    &= y'_1 u'_1 + y'_2 u'_2,
\end{align*}

because of the assumption $u'_1 y_1 + u'_2 y_2  = 0$.

\end{frame}

\begin{frame}\frametitle{Putting it all together}

We have shown that if $u_1$ and $u_2$ have the desired properties, then
they satisfy

\begin{align*}
    u'_1 y'_1 + u'_2 y'_2 &= g(t) \\
    u'_1 y_1 + u'_2 y_2  &= 0.
\end{align*}

Solving this system of equations for $u_1$ and $u_2$ (in the usual
algebraic way, by either substitution or addition of equations) gives

\begin{equation*}
    u'_1 = -\frac{y_2 g}{W(y_1, y_2)}, u'_2 = \frac{y_1 g}{W(y_1, y_2)}.
\end{equation*}

Integrating each of these yields the desired functions $u_1$ and $u_2$.

\end{frame}

\section{Conclusion}

\begin{frame}\frametitle{The particular solution}

That is, a particular solution $Y$ results from any choice of
antiderivatives $u_1$, $u_2$. We have

\begin{equation*}
    Y = -y_1 \int \frac{y_2 g}{W(y_1, y_2)} \; dt + y_2 \int \frac{y_1 g}{W(y_1, y_2)} \; dt.
\end{equation*}

Like undetermined coefficients, this method has its own characteristics.

\begin{itemize}
\itemsep1pt\parskip0pt\parsep0pt
\item
  It is reassuringly mechanistic. $Y$ is given by a formula.
\item
  It is also very general, since there are no conditions on $g$.
\item
  On the other hand, the integrals that arise may be intractable.
\end{itemize}

\end{frame}

\begin{frame}\frametitle{Generalizations}

For us, the coefficient functions $q(t)$ and $r(t)$ are constant. What
happens when they are not?

\begin{itemize}
\itemsep1pt\parskip0pt\parsep0pt
\item
  The methods of Chapter 5 are required to solve the general
  second-order linear homogeneous equation.
\item
  If fundamental solutions are known, VP works just the same to
  determine a particular solution.
\item
  It is only the characteristic equation and the exponential trick that
  fail in this case.
\end{itemize}

\end{frame}

\end{document}
