\def\encoding{UTF-8}
\input{mmd-beamer-header-rosoff}
\def\mytitle{Variation of parameters}
\def\affiliation{The College of Idaho}
\def\myauthor{Math 352 Differential Equations}
\def\mydate{8 April 2013}
\def\latexmode{beamer}
\input{mmd-beamer-begin-doc-rosoff}
\def\htmlheaderlevel{2}
\section{Introduction}
\label{introduction}

\begin{frame}

\frametitle{Introduction: This one goes out to all the $ g $s}
\label{introduction:thisonegoesouttoallthegs}

The method of undetermined coefficients is very useful when the derivatives of the terms appearing in $ g(t) $ have a regular, predictable shape and are themselves the derivatives of similar functions. But it's not too hard to find functions $ g(t) $ that would demand excessive ingenuity of us when guessing the form of $ Y $.

\end{frame}

\begin{frame}

\frametitle{Introduction: Too many functions!}
\label{introduction:toomanyfunctions}

Consider $ y'' + y' + y = 3 \csc{(t)} $. You can check that a particular solution $ Y $ is given by the formula
 \pause 
\[
    -3 \sin{(t)} \cos{(2t)} + \frac{3}{2} \ln |\csc{(t)} - \cot{(t)}| \sin{(2t)} + 3 \cos{(t)} \sin{(2t)}.
\]
 \pause 
It's hard to imagine rules for undetermined coefficients that are both

\begin{itemize}
\item comprehensive enough to include a function like this and

\item possible to commit to memory.

\end{itemize}

\end{frame}

\section{Variation of parameters}
\label{variationofparameters}

\begin{frame}

\frametitle{Variation of parameters: not a new thought}
\label{variationofparameters:notanewthought}

Recall how we discovered the second fundamental solution $ t e^{rt} $ in the case of a repeated root $ r $ of the characteristic equation. We promoted a constant to a functional coefficient and looked for a solution of the form $ v(t) \exp{(rt)} $.

\begin{itemize}
\item In other words, we allowed a parameter to vary.

\end{itemize}

We can try a similar idea when tackling the inhomogeneous equation $ y'' + q(t)y' + r(t)y = g(t) $, provided we have already solved the associated homogeneous equation $ y'' + q(t)y' + r(t)y = 0 $. Let us write $ y_1 $ and $ y_2 $ for a fundamental set of solutions as usual.

\end{frame}

\begin{frame}

\frametitle{Variation of parameters: execution}
\label{variationofparameters:execution}

We shall search for a particular solution $ Y $ of our inhomogeneous equation of the form $ Y = u_1 y_1 + u_2 y_2 $: that is to say,
\[
    Y(t) = u_1(t) y_1(t) + u_2(t) y_2(t),
\]
where $ u_1 $ and $ u_2 $ are unknown coefficient functions we will have to find. We'll be plugging this expression for $ Y $ along with its derivatives back into the inhomogeneous equation, so let's compute the derivatives now.

\end{frame}

\begin{frame}

\frametitle{An unjustified assumption}
\label{anunjustifiedassumption}

For no very good reason, let's assume that the fundamental solutions $ y_1 $ and $ y_2 $ and the derivatives $ u'_1 $ and $ u'_2 $ satisfy the following equation:
\[
    u'_1 y_1 + u'_2 y_2  = 0.
\]

\begin{itemize}
\item This is the worst rabbit-out-of-the-hat of the term.

\item Sorry.

\item The idea is evidently due to Lagrange\footnote{Joseph-Louis Lagrange (25 January 1736--10 April 1813) is most remembered for contributions analysis, number theory, and classical and celestial mechanics. His \emph{M\'ecanique Analytique} was fundamental for the physicists of the 19th century.}, and it greatly simplifies the computation to come.

\item At no point does it cause problems of any kind.

\end{itemize}

\end{frame}

\begin{frame}

\frametitle{The derivatives of $ Y $}
\label{thederivativesofy}

Since
\[
    Y(t) = u_1(t) y_1(t) + u_2(t) y_2(t),
\]
we also have

\begin{align*}
    Y'  &= u'_1 y_1 + u_1 y'_1 + u'_2 y_2 + u_2 y'_2, \\
        &= u_1 y'_1 + u_2 y'_2, \quad \text{and so} \\
    Y'' &= u_1 y''_1 + u'_1 y'_1 + u'_2 y'_2 + u_2 y''_2.
\end{align*}


\end{frame}

\begin{frame}

\frametitle{Substitute back in}
\label{substitutebackin}

Now we find that $ Y''(t) + q(t)Y'(t) + r(t)Y(t) $ reduces to
\[
    (u_1 y''_1 + u'_1 y'_1 + u'_2 y'_2 + u_2 y''_2) + q(t)(u_1 y'_1 + u_2 y'_2) + r(t)(u_1 y_1 + u_2 y_2).
\]
Collecting terms along $ u_1 $, $ u_2 $, and their derivatives, we get
 
\begin{align*}
    (y''_1  & + qy'_1 + ry_1)u_1 + (y''_2 + qy'_2 + ry_2)u_2 + y'_1 u'_1 + y'_2 u'_2 \\
    &= y'_1 u'_1 + y'_2 u'_2,
\end{align*}
 
because of the assumption $ u'_1 y_1 + u'_2 y_2  = 0 $.

\end{frame}

\begin{frame}

\frametitle{Putting it all together}
\label{puttingitalltogether}

We have shown that if $ u_1 $ and $ u_2 $ have the desired properties, then they satisfy
 \begin{align*}
    u'_1 y'_1 + u'_2 y'_2 &= g(t) \\
    u'_1 y_1 + u'_2 y_2  &= 0.
\end{align*}
 
Solving this system of equations for $ u_1 $ and $ u_2 $ (in the usual algebraic way, by either substitution or addition of equations)
gives
\[
    u'_1 = -\frac{y_2 g}{W(y_1, y_2)}, u'_2 = \frac{y_1 g}{W(y_1, y_2)}.
\]
Integrating each of these yields the desired functions $ u_1 $ and $ u_2 $.

\end{frame}

\section{Conclusion}
\label{conclusion}

\begin{frame}

\frametitle{The particular solution}
\label{theparticularsolution}

That is, a particular solution $ Y $ results from any choice of antiderivatives $ u_1 $, $ u_2 $. We have
\[
    Y = -y_1 \int \frac{y_2 g}{W(y_1, y_2)} \; dt + y_2 \int \frac{y_1 g}{W(y_1, y_2)} \; dt.
\]
Like undetermined coefficients, this method has its own characteristics.

\begin{itemize}
\item It is reassuringly mechanistic. $ Y $ is given by a formula.

\item It is also very general, since there are no conditions on $ g $.

\item On the other hand, the integrals that arise may be intractable.

\end{itemize}

\end{frame}

\begin{frame}

\frametitle{Generalizations}
\label{generalizations}

For us, the coefficient functions $ q(t) $ and $ r(t) $ are constant. What happens when they are not?

\begin{itemize}
\item The methods of Chapter 5 are required to solve the general second-order linear homogeneous equation.

\item If fundamental solutions are known, VP works just the same to determine a particular solution.

\item It is only the characteristic equation and the exponential trick that fail in this case.

\end{itemize}

\end{frame}

\mode<all>
\input{mmd-beamer-footer}

\end{document}\mode*

