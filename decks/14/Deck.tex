\documentclass[11pt,ignorenonframetext,]{beamer}
\usetheme{Szeged}
\usecolortheme{wolverine}
\usefonttheme{structurebold}
\usepackage{amssymb,amsmath}
\usepackage{ifxetex,ifluatex}
\usepackage{fixltx2e} % provides \textsubscript
\ifxetex
  \usepackage{fontspec,xltxtra,xunicode}
  \defaultfontfeatures{Mapping=tex-text,Scale=MatchLowercase}
\else
  \ifluatex
    \usepackage{fontspec}
    \defaultfontfeatures{Mapping=tex-text,Scale=MatchLowercase}
  \else
    \usepackage[utf8]{inputenc}
  \fi
\fi
\usepackage{listings}

% Comment these out if you don't want a slide with just the
% part/section/subsection/subsubsection title:
% \AtBeginPart{
%   \let\insertpartnumber\relax
%   \let\partname\relax
%   \frame{\partpage}
% }
% \AtBeginSection{
%   \let\insertsectionnumber\relax
%   \let\sectionname\relax
%   \frame{\sectionpage}
% }
% \AtBeginSubsection{
%   \let\insertsubsectionnumber\relax
%   \let\subsectionname\relax
%   \frame{\subsectionpage}
% }

\setlength{\parindent}{0pt}
\setlength{\parskip}{6pt plus 2pt minus 1pt}
\setlength{\emergencystretch}{3em}  % prevent overfull lines
\setcounter{secnumdepth}{0}

%%% begin dwr insert
\usepackage{patchcmd}
\usepackage{tabulary}   % Support longer table cells
\usepackage{booktabs}   % Support better tables
\usepackage[sort&compress]{natbib}

\usepackage{framed}     % Allow background color for images
\definecolor{shadecolor}{named}{white}

%\usepackage{paralist}
\usepackage{xparse}
\usepackage{subfigure}
\usepackage{hyperref}
%%% end dwr insert
\usepackage[beamer]{rosoff}
\title{Title}
\author{Math 352 Differential Equations}
\date{April 4, 2014}


\begin{document}
\frame{\titlepage}

\section{Introduction}

\begin{frame}\frametitle{The machinery of linear algebra}

\begin{itemize}
\item
  Fundamental algebraic construction of linear algebra: \emph{linear
  combinations}.
\item
  \emph{Vector spaces} are places where it makes sense to form linear
  combos.
\item
  Coefficients can be real, complex, \ldots{}
\item
  The objects being combined are often called vectors, even if they are
  not vectors like $\langle 2, -3, 4, 12 \rangle$ or
  $2 \vec{i} - 6 \vec{j}$.
\item
  Suppose that $y_1$, $y_2$, $y_3$, \ldots{} $y_n$ are differentiable
  and that $a_1$, \ldots{} $a_n$ are real numbers (scalars).
\item
  Then $a_1 y_1 + a_2 y_2 + \cdots + a_n y_n$ is also differentiable.
\item
  So, the set of differentiable functions forms what is called a
  \emph{real vector space}.
\end{itemize}

\end{frame}

\begin{frame}\frametitle{Linear systems}

A system of linear (algebraic) equations is a system like this:

\begin{equation*}
    \begin{array}{ccccccccc}
        a_{11} x_{1} &+& a_{12} x_{2} &+& a_{13} x_{3} &+& a_{14} x_{4} &=& b_1 \\
        a_{21} x_{1} &+& a_{22} x_{2} &+& a_{23} x_{3} &+& a_{24} x_{4} &=& b_2 \\
        a_{31} x_{1} &+& a_{32} x_{2} &+& a_{33} x_{3} &+& a_{34} x_{4} &=& b_3 \\
    \end{array}
\end{equation*}

For example, initial value problems give rise to such systems, as did
undetermined coefficients:

\begin{equation*}
    \begin{array}{ccccc}
        c_1   &+&  c_2 &=& 2 \\
        -4c_1 &+& 2c_2 &=& -10 \\
    \end{array}
\end{equation*}

\end{frame}

\begin{frame}\frametitle{Matrix form of a system}

\begin{equation*}
    \begin{pmatrix}
        a_{11} & a_{12} & a_{13} & a_{14} \\
        a_{21} & a_{22} & a_{23} & a_{24} \\
        a_{31} & a_{32} & a_{33} & a_{34} \\
    \end{pmatrix} \begin{pmatrix}
        x_1 \\ x_2 \\ x_3 \\ x_4
    \end{pmatrix} = \begin{pmatrix}
        b_1 \\ b_2 \\ b_3
    \end{pmatrix}
\end{equation*}

It's cleaner and allows us to focus on the arithmetic.

\begin{equation*}
    \begin{pmatrix}
        1 & 1 \\
        -4 & 2 \\ 
    \end{pmatrix} \begin{pmatrix}
        x_1 \\ x_2
    \end{pmatrix} = \begin{pmatrix}
        2 \\ -10
    \end{pmatrix}
\end{equation*}

\end{frame}

\begin{frame}\frametitle{Solutions of matrix equations}

\begin{itemize}
\itemsep1pt\parskip0pt\parsep0pt
\item
  The vector of $x_i$ is the variable of the matrix equation.
\item
  A \emph{solution} of the matrix equation is a choice of $x_i$ that
  makes all the individual linear equations true.
\end{itemize}

Two matrix equations are \emph{equivalent} if they have exactly the same
set of solutions.

\begin{itemize}
\itemsep1pt\parskip0pt\parsep0pt
\item
  Matrices are solved by transforming them into equivalent equations
  whose solutions are obvious.
\end{itemize}

\end{frame}

\begin{frame}\frametitle{An obvious matrix}

Lots of other matrix equations' solutions are obvious, but this is what
I really meant:

\begin{equation*}
    \begin{pmatrix}
        \textcolor{red}{1}  & 2 & \textcolor{blue}{0} & \textcolor{blue}{0} & 2/3  \\
        \textcolor{blue}{0} & 0 & \textcolor{red}{1}  & \textcolor{blue}{0} & -1  \\
        \textcolor{blue}{0} & 0 & \textcolor{blue}{0} & \textcolor{red}{1} & -10  \\
        \textcolor{blue}{0} & 0 & \textcolor{blue}{0} & \textcolor{blue}{0} &  0  \\
    \end{pmatrix} \begin{pmatrix}
        x_1 \\ x_2 \\ x_3 \\ x_4 \\ x_5 
    \end{pmatrix} = \begin{pmatrix}
        b_1 \\ b_2 \\ b_3 \\ b_4
    \end{pmatrix}
\end{equation*}

\end{frame}

\begin{frame}\frametitle{Reduced echelon form}

We say a matrix $A$ is in reduced echelon form if:

\begin{itemize}
\itemsep1pt\parskip0pt\parsep0pt
\item
  the first nonzero entry of each row is a $1$

  \begin{itemize}
  \itemsep1pt\parskip0pt\parsep0pt
  \item
    such an entry is called a \texttt{pivot'' or}leading $1$''
  \end{itemize}
\item
  each pivot is the only nonzero entry in its column
\item
  each of the pivots after the first one appears to the right of the
  previous pivot
\item
  each row without a pivot follows all rows with a pivot
\end{itemize}

Such a matrix certainly has an obvious solution set, and:

\begin{itemize}
\itemsep1pt\parskip0pt\parsep0pt
\item
  Every matrix is equivalent to exactly one matrix in reduced echelon
  form.
\end{itemize}

\end{frame}

\begin{frame}\frametitle{How do we find a reduced echelon equivalent?}

\begin{itemize}
\item
  The same way we solve the equations to which the matrix equation
  corresponds: by adding and subtracting the rows from one another.
\item
  We'll need to keep track of the RHS too, so add it as the last column
  of the matrix. We'll manipulate this matrix:
\end{itemize}

\begin{equation*}
    \begin{pmatrix}
        a_{11} & a_{12} & \cdots & a_{1n} & b_1 \\
        a_{21} & a_{22} & \cdots & a_{2n} & b_2 \\
        \vdots & \vdots & \ddots & \vdots & \vdots \\
        a_{m1} & a_{m2} & \cdots & a_{mn} & b_m \\
    \end{pmatrix}
\end{equation*}

\end{frame}

\begin{frame}\frametitle{Gaussian elimination}

\begin{itemize}
\item
  There are three operations on matrices that preserve solution sets.
\item
  If you do any of these operations to a matrix, you obtain an
  equivalent one.
\item
  Swapping rows

  \begin{itemize}
  \itemsep1pt\parskip0pt\parsep0pt
  \item
    obviously, this won't affect the solutions: underlying set of
    equations is the same
  \end{itemize}
\item
  Multiplying a row by a nonzero number

  \begin{itemize}
  \itemsep1pt\parskip0pt\parsep0pt
  \item
    this changes the equations, but not the solution set
  \end{itemize}
\item
  Adding a nonzero multiple of a row to another row

  \begin{itemize}
  \itemsep1pt\parskip0pt\parsep0pt
  \item
    also doesn't change the solution set.
  \end{itemize}
\end{itemize}

There's a fairly transparent algorithm using these operations that
transforms each matrix into its unique reduced echelon equivalent.

\end{frame}

\begin{frame}\frametitle{In practice: upper-triangular}

If you are row-reducing a matrix by hand, reduced echelon form is
overkill a lot of the time.

\begin{itemize}
\itemsep1pt\parskip0pt\parsep0pt
\item
  Reduce to an upper-triangular matrix
\item
  All nonzero entries on or above the main diagonal

  \begin{itemize}
  \itemsep1pt\parskip0pt\parsep0pt
  \item
    that is, the upper-left-to-lower-right diagonal
  \item
    with slope $-1$
  \end{itemize}
\end{itemize}

\end{frame}

\begin{frame}\frametitle{Entering vectors in Sage}

\begin{itemize}
\itemsep1pt\parskip0pt\parsep0pt
\item
  Initialize a vector value with \texttt{vector()}
\item
  \texttt{u = vector(QQ, [1, 3/2, -1])}
\item
  Use \texttt{QQ} to display fractions instead of decimals
\item
  \texttt{u}

  \begin{itemize}
  \itemsep1pt\parskip0pt\parsep0pt
  \item
    \texttt{(1, -3/2, 1)}
  \end{itemize}
\item
  \texttt{v = vector(RR, [1, 3/2, -1])}; v

  \begin{itemize}
  \itemsep1pt\parskip0pt\parsep0pt
  \item
    \texttt{(1.00000000000000, 1.50000000000000, -1.00000000000000)}
  \end{itemize}
\end{itemize}

\end{frame}

\begin{frame}\frametitle{Entering matrices}

\begin{itemize}
\itemsep1pt\parskip0pt\parsep0pt
\item
  Either enter a matrix as a vector of its rows

  \begin{itemize}
  \itemsep1pt\parskip0pt\parsep0pt
  \item
    \texttt{A = matrix(QQ, [[1, 2], [3, 4], [5, 6]])}
  \end{itemize}
\item
  or as a list with specification of number of rows

  \begin{itemize}
  \itemsep1pt\parskip0pt\parsep0pt
  \item
    \texttt{A = matrix(QQ, 2, [1,2,3,4,5,6])}
  \end{itemize}
\item
  Obtain reduced echelon form with \texttt{rref}:

  \begin{itemize}
  \itemsep1pt\parskip0pt\parsep0pt
  \item
    \texttt{B = A.rref()}
  \end{itemize}
\end{itemize}

\end{frame}

\begin{frame}\frametitle{Row operations}

\begin{itemize}
\itemsep1pt\parskip0pt\parsep0pt
\item
  Want to use Sage to check your work in performing row operations?

  \begin{itemize}
  \itemsep1pt\parskip0pt\parsep0pt
  \item
    The matrix methods below may be useful.
  \item
    These methods are \emph{destructive}: they change the entries of the
    matrix on which they're called.
  \end{itemize}
\item
  \texttt{A.rescale_row(i,a)}

  \begin{itemize}
  \itemsep1pt\parskip0pt\parsep0pt
  \item
    multiply row $i$ by $a$
  \end{itemize}
\item
  \texttt{A.add_multiple_of_row(i,j,a)}

  \begin{itemize}
  \itemsep1pt\parskip0pt\parsep0pt
  \item
    add $a$ times row $j$ to row $i$
  \end{itemize}
\item
  \texttt{A.swap_rows(i,j)}

  \begin{itemize}
  \itemsep1pt\parskip0pt\parsep0pt
  \item
    swap rows $i$ and $j$
  \end{itemize}
\end{itemize}

\end{frame}

\begin{frame}\frametitle{Sage lab assignment}

Use Sage to solve the systems of linear equations.

\begin{equation*}
    \begin{array}{ccccccc}
        3x &+& 3y &+& 12z &=& 6 \\
        x  &+& y  &+& 4z  &=& 2 \\
        2x &+& 5y &+& 20z &=& 10 \\
        -x &+& 2y &+& 8z  &=& 4  
    \end{array} \quad \begin{array}{ccccccc}
        2x  &+& 10y &+& 2z &=& 6 \\
        x   &+& 5y  &+& 2z &=& 6 \\
        x   &+& 5y  &+& z  &=& 3 \\
        -3x &-& 15y &+& 3z &=& -9  
    \end{array}
\end{equation*}

\begin{equation*}
    \begin{array}{ccccccccc}
        2x &+& y &-& z &+& 2w &=& -6 \\
        3x &+& 4y & & &+& w &=& 1 \\
        x &+& 5y &+& 2z &+& 6w &=& -3 \\
        5x &+& 2y &-& z &-& w &=& 3
    \end{array}
\end{equation*}

\end{frame}

\end{document}
