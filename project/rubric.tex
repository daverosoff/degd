\documentclass{article}
\usepackage[margin=0.5in]{geometry}
\usepackage{fourier, rotating,tabularx, booktabs}
\usepackage{hyperref}
\title{Grading Rubric for Math 352 Projects}
\pagestyle{empty}
\begin{document}
\thispagestyle{empty}
\maketitle
\begin{minipage}[t]{0.47\textwidth}
\vspace{0pt}
Project title: \enspace\underline{\hfill} \\[5.0ex]
Group member name: \enspace\underline{\hfill} 
\end{minipage} \hspace{0.5cm}
\begin{minipage}[t]{0.47\textwidth}
\vspace{0pt}
Group member name: \enspace\underline{\hfill} \\[1.0ex]
Group member name: \enspace\underline{\hfill} \\[1.0ex]
Group member name: \enspace\underline{\hfill}
\end{minipage}

\vspace{\stretch{1}}

  \begin{tabularx}{\linewidth}{XXXXX} \toprule \\
  & 20 points & 15 points & 10 points & 5 points \\[1.2ex] \midrule
  \textbf{Description of the model (20 points)} 
  & The description of the model is well-motivated, convincing, 
  clearly conveyed, and tailored to the intended audience of the paper.
  & The model is described well, but is missing one or two 
  things that the audience would need to know to fully understand.
  & The model is hastily described or is missing significant 
  details.
  & The model is ill-defined. \\[1.2ex] 
  \textbf{Problem formulation (20 points)} 
  & The model is accurately 
  formulated: all assumptions are clearly stated and warranted. 
  & Some unwarranted assumptions were made in the model.
  & There are some flaws in the formulation of the model. 
  & The model is either largely incorrect or inappropriate for the 
  problem. \\[1.2ex] 
  \textbf{Quality of mathematical work (20 points)} 
  & The analysis techniques chosen are well-suited to the problem. The 
  calculations performed are thorough, correct, and complete. 
  & The analysis techniques work well for the problem, but there may be 
  other techniques that are superior. The calculations are mostly complete 
  and correct.
  & The analysis techniques chosen are not well implemented or inappropriate.
  Some calculations are incorrect.
  & There are serious mathematical errors that lead to incorrect results. 
  \\[1.2ex] 
  \textbf{Insightfulness and depth of understanding (20 points)} 
  & Numerous insightful and thoughtful conclusions from data are presented. 
  The authors designed the model and calculations well to 
  reach these conclusions and they learned  many things from 
  their model. 
  & Some insightful conclusions are reached through the model and the 
  calculations. There are a few obvious things the authors could have done 
  to draw more conclusions from the model.
  & The authors make a cursory attempt to draw conclusions from the data. 
  The authors demonstrate a minimal advancement of learning ability through 
  the model.
  & The write-up gives no insights on the original problem or shows no signs 
  of learning on the part of the authors. \\[1.2ex] 
  \textbf{Grammar and mechanics (20 points)} 
  &  Excellent. The paper is easy to read, with a clear logical structure
  and few or no misspelled words or other grammatical or typographical 
  errors.
  &  Fair. The logical structure of the paper and the argument is mostly
  apparent. While there are some errors, there are not too many and they
  distract from the argument only a little.
  &  Borderline. The argument is haphazardly constructed, with only minimal
  logical structure. There are many usage errors.
  &  Poorly written. There seems to be no attempt to make the paper coherent
  as a whole. A great multitude of errors make it unreadable. \\[1.2ex] 
  \bottomrule
  \end{tabularx}
\end{document}