\documentclass[12pt,twoside]{exam}
\usepackage[utf8]{inputenc}
\usepackage{rosoff}
\usepackage{xparse}
\usepackage{graphicx}
\DeclareGraphicsExtensions{.jpg, .png}
\usepackage{fourier}
%\usepackage{amsthm}
\usepackage{listings,booktabs,tabularx}
%\usepackage[inline]{enumitem}
%\usepackage{siunitx}
\frenchspacing
\usepackage{parskip}
\usepackage{pgfplots}
\usepackage{rosoff}
\usepackage{hyperref}
\firstpageheader{}{}{}
\runningheader{\textbf{Spring 2014}}
 {}
 {\textbf{Math 352}}
 %{\emph{Page \thepage~of \numpages}}
\runningheadrule

\pagestyle{head}
\extrawidth{1in}
\extraheadheight[-0.5in]{0in}
\extrafootheight{-0.5in}
\begin{document}
\noindent
\textbf{{\large Math 352 \hfill Workshop 13}}
% \hfill Name: \underline{\hspace{0.5in}Answers\hspace{2in}}

\vspace{2ex}

\noindent
\makebox[\textwidth]{April 28, 2014 \hfill Due: Monday, April 28 \hfill Name: \underline{\hspace{3in}} }

\noindent

\newcommand{\longlines}{\setlength{\answerlinelength}{0.7\linewidth}}
\newcommand{\medlines}{\setlength{\answerlinelength}{0.45\linewidth}}
\newcommand{\shortlines}{\setlength{\answerlinelength}{0.2\linewidth}}

\RenewDocumentCommand\vec{m}{\ensuremath{\mathbf{#1}}}

\subsection*{Plotting solutions in the phase plane}

Sage commands that will be useful include

\texttt{plot\_vector\_field()}, \texttt{parametric\_plot()}, and
\texttt{A.eigenspaces\_right()} (where \texttt{A} is a square matrix).

Remember, you can use the \texttt{help()} command to get information on
any of these commands.

Last time, we saw that if the $2 \times 2$ matrix $A$ has real
eigenvalues $r_1$ and $r_2$ with corresponding eigenvectors
$\vec{\xi}^{(1)}$ and $\vec{\xi}^{(2)}$, then

\begin{equation*}
    \vec{\xi}^{(1)} e^{r_1 t}, \quad \vec{\xi}^{(2)} e^{r_2 t}
\end{equation*}

are (vector-valued) solutions of the homogeneous system
$\vec{x}' = A\vec{x}$.

\begin{questions}

    \question For each of the $2 \times 2$ systems, use Sage
    to plot solutions in the phase plane. Make some notes about
    how the solutions appear. The last two will be rather different
    from the others.

    \begin{parts}
    \begin{minipage}[t]{0.47\textwidth}
            \part {\begin{flalign*}
                            x'_1 &= 3x_1 - 2x_2 & \\
                            x'_2 &= 2x_1 - 2x_2 &
                        \end{flalign*}} 

            \part {\begin{flalign*}
                            x'_1 &= x_1 - 2x_2 &  \\
                            x'_2 &= 3x_1 - 4x_2 &
                        \end{flalign*} }

            \part \begin{flalign*}
                x'_1 &= 2x_1 - x_2  & \\
                x'_2 &= 3x_1 - 2x_2 &
            \end{flalign*} 

            \part \begin{flalign*}
                x'_1 &= x_1 + x_2  & \\
                x'_2 &= 4x_1 - 2x_2 &
            \end{flalign*} 
    \end{minipage}
    \begin{minipage}[t]{0.47\textwidth}
            \part \begin{flalign*}
                x'_1 &= -2x_1 + x_2  & \\
                x'_2 &= x_1 - 2x_2 &
            \end{flalign*}

            \part \begin{flalign*}
                x'_1 &= \frac{5}{4} x_1 + \frac{3}{4} x_2  & \\
                x'_2 &= \frac{3}{4} x_1 + \frac{5}{4} x_2 &
            \end{flalign*}

            \part \begin{flalign*}
                x'_1 &= 4x_1 - 3x_2   &\\
                x'_2 &= 8x_1 - 6x_2 &
            \end{flalign*}

            \part \begin{flalign*}
                x'_1 &= 3x_1 + 6x_2  & \\
                x'_2 &= -x_1 - 2x_2 &
            \end{flalign*} 
    \end{minipage}
    \end{parts}
               
\end{questions} 

\end{document}