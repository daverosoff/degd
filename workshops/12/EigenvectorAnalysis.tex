\documentclass[12pt,twoside]{exam}
\usepackage[utf8]{inputenc}
\usepackage{rosoff}
\usepackage{xparse}
\usepackage{graphicx}
\DeclareGraphicsExtensions{.jpg, .png}
\usepackage{fourier}
%\usepackage{amsthm}
\usepackage{listings,booktabs,tabularx}
%\usepackage[inline]{enumitem}
%\usepackage{siunitx}
\frenchspacing
\usepackage{parskip}
\usepackage{pgfplots}
\usepackage{rosoff}
\usepackage{hyperref}
\firstpageheader{}{}{}
\runningheader{\textbf{Spring 2014}}
 {}
 {\textbf{Math 352}}
 %{\emph{Page \thepage~of \numpages}}
\runningheadrule

\pagestyle{head}
\extrawidth{1in}
\extraheadheight[-0.5in]{0in}
\extrafootheight{-0.5in}
\begin{document}
\noindent
\textbf{{\large Math 352 \hfill Workshop 12}}
% \hfill Name: \underline{\hspace{0.5in}Answers\hspace{2in}}

\vspace{2ex}

\noindent
\makebox[\textwidth]{April 25, 2014 \hfill Due: Monday, April 28 \hfill Name: \underline{\hspace{3in}} }

\noindent

\newcommand{\longlines}{\setlength{\answerlinelength}{0.7\linewidth}}
\newcommand{\medlines}{\setlength{\answerlinelength}{0.45\linewidth}}
\newcommand{\shortlines}{\setlength{\answerlinelength}{0.2\linewidth}}

%\RenewDocumentCommand\vec{m}{\ensuremath{\mathbf{#1}}}

\subsection*{Eigenvectors, phase portraits, and stability}

Sage commands that will be useful include

\texttt{plot\_vector\_field()}, \texttt{parametric\_plot()}, and
\texttt{A.eigenspaces\_right()} 

(where \texttt{A} is a square matrix).
Remember, you can use the \texttt{help()} command to get information on
any of these commands, e.g. \texttt{help(parametric\_plot)}.

Last time, we saw that if the $2 \times 2$ matrix $A$ has real
eigenvalues $r_1$ and $r_2$ with corresponding eigenvectors
$\vec{\xi}^{(1)}$ and $\vec{\xi}^{(2)}$, then

\begin{equation*}
    \vec{\xi}^{(1)} e^{r_1 t}, \quad \vec{\xi}^{(2)} e^{r_2 t}
\end{equation*}

are (vector-valued) solutions of the homogeneous system
$\vec{x}' = A\vec{x}.$

\begin{questions}

    \question Explain in one sentence, with no mathematical symbols, what
    it means for a vector to be an eigenvector of some matrix.

    \question List the eigenvalues and eigenvectors of the following
    matrices. Make sure you preserve the correspondence between
    an eigenvalue and its eigenvector(s).

    \begin{parts}
        \begin{tabularx}{\linewidth}{XX}
            \part $\begin{pmatrix}
                1 & 2 \\ 0 & -3 
            \end{pmatrix}$ &
            \part $\begin{pmatrix}
                2 & 0 \\ 0 & -1
            \end{pmatrix}$ \\[1.0in]
            \part $\begin{pmatrix}
                2 & 3 \\ 3 & -6          
            \end{pmatrix}$ &
            \part $\begin{pmatrix}
                -6 & 2 \\ 3 & -1
            \end{pmatrix}$ \\[1.0in]
            \part $\begin{pmatrix}
                1 & -2 & 1 \\
                0 & 0 & 0 \\
                0 & 1 & 1        
            \end{pmatrix}$ &
            \part $\begin{pmatrix}
                1 & 3 & 0 \\
                3 & 1 & 0 \\
                0 & 0 & -2        
            \end{pmatrix}$ \\[1.0in]
        \end{tabularx}
        \newpage
        \begin{tabularx}{\linewidth}{XX}
            \part $\begin{pmatrix}
                1 & 0 & 0 & 0\\
                0 & 1 & 5 & -10 \\
                1 & 0 & 2 & 0 \\
                1 & 0 & 0 & 3         
            \end{pmatrix}$ &
            \part $\begin{pmatrix}
                2 & 1 & 0 \\
                0 & 2 & 0 \\
                0 & 0 & 2         
            \end{pmatrix}$ \\[1.0in]
        \end{tabularx}
    \end{parts}

    \question For each of the matrices $A$ above, write down all the
    solutions of exponential type (of $\vec{x}' = A \vec{x}$) you can find.

    \begin{parts}
        \begin{tabularx}{\linewidth}{XX}
            \part $\begin{pmatrix}
                1 & 2 \\ 0 & -3 
            \end{pmatrix}$ &
            \part $\begin{pmatrix}
                2 & 0 \\ 0 & -1
            \end{pmatrix}$ \\[1.0in]
            \part $\begin{pmatrix}
                2 & 3 \\ 3 & -6          
            \end{pmatrix}$ &
            \part $\begin{pmatrix}
                -6 & 2 \\ 3 & -1
            \end{pmatrix}$ \\[1.0in]
            \part $\begin{pmatrix}
                1 & -2 & 1 \\
                0 & 0 & 0 \\
                0 & 1 & 1        
            \end{pmatrix}$ &
            \part $\begin{pmatrix}
                1 & 3 & 0 \\
                3 & 1 & 0 \\
                0 & 0 & -2        
            \end{pmatrix}$ \\[1.5in]
        % \end{tabularx}
        % \newpage
        % \begin{tabularx}{\linewidth}{XX}
            \part $\begin{pmatrix}
                1 & 0 & 0 & 0\\
                0 & 1 & 5 & -10 \\
                1 & 0 & 2 & 0 \\
                1 & 0 & 0 & 3         
            \end{pmatrix}$ &
            \part $\begin{pmatrix}
                2 & 1 & 0 \\
                0 & 2 & 0 \\
                0 & 0 & 2         
            \end{pmatrix}$ \\[1.5in]
        \end{tabularx}
    \end{parts}


\end{questions} 

\end{document}