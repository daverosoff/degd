\documentclass[12pt,twoside]{exam}
\usepackage[utf8]{inputenc}
\usepackage{rosoff}
\usepackage{xparse}
\usepackage{graphicx}
\DeclareGraphicsExtensions{.jpg, .png}
\usepackage{fourier}
%\usepackage{amsthm}
\usepackage{listings,booktabs,tabularx}
%\usepackage[inline]{enumitem}
%\usepackage{siunitx}
\frenchspacing
\usepackage{parskip}
\usepackage{pgfplots}
\usepackage{rosoff}
\usepackage{hyperref}
\firstpageheader{}{}{}
\runningheader{\textbf{Spring 2014}}
 {}
 {\textbf{Math 352}}
 %{\emph{Page \thepage~of \numpages}}
\runningheadrule

\pagestyle{head}
\extrawidth{1in}
\extraheadheight[-0.5in]{0in}
\extrafootheight{-0.5in}
\begin{document}
\noindent
\textbf{{\large Math 352 \hfill Workshop 14}}
% \hfill Name: \underline{\hspace{0.5in}Answers\hspace{2in}}

\vspace{2ex}

\noindent
\makebox[\textwidth]{May 2, 2014 \hfill Not collected \hfill Name: \underline{\hspace{2in}} }

\noindent

\newcommand{\longlines}{\setlength{\answerlinelength}{0.7\linewidth}}
\newcommand{\medlines}{\setlength{\answerlinelength}{0.45\linewidth}}
\newcommand{\shortlines}{\setlength{\answerlinelength}{0.2\linewidth}}

\RenewDocumentCommand\vec{m}{\ensuremath{\mathbf{#1}}}

\subsection*{Complex eigenvalues, limit cycles, spiral points}

Sage commands that will be useful include

\texttt{plot\_vector\_field()}, \texttt{parametric\_plot()}, and
\texttt{A.eigenspaces\_right()} 

(where \texttt{A} is a square matrix).
Remember, you can use the \texttt{help()} command to get information on
any of these commands, e.g. \texttt{help(parametric\_plot)}.

\begin{questions}

    \question For each of the systems of equations below, complete
    the following steps.
    \begin{compactenum}[i.]
        \item Use the method of eigenvectors to find the general 
        (real-valued) solution of the system.
        \item Use Sage to make a phase plot that shows three trajectories.
        \item Make note of any limit cycles or equilibrium solutions.
    \end{compactenum}

    \begin{parts}
        \begin{minipage}[t]{0.47\linewidth}
        \vspace{0pt}
            \part \begin{flalign*}
                x'_1 &= x_2  &\\
                x'_2 &= x_1 &
            \end{flalign*} 

            \vspace{2in}

            \part \begin{flalign*}
                x'_1 &= 3x_1 - 2x_2  &\\
                x'_2 &= 4x_1 - x_2 &
            \end{flalign*} 

            \vspace{2in}

        \end{minipage}
        \begin{minipage}[t]{0.47\linewidth}
        \vspace{0pt}
            \part \begin{flalign*}
              x'_1 &= 2x_1  -(5/2) x_2  &\\
              x'_2 &= (9/5) x_1  -x_2 &
            \end{flalign*} 

            \vspace{2in}

            \part \begin{flalign*}
              x'_1 &= 2x_1 - 5x_2  &\\
              x'_2 &= x_1 - 2x_2 &
            \end{flalign*}   

            \vspace{2in} 
        \end{minipage}      
        
    \end{parts}
    
    \newpage

    \question These systems contain a parameter $\alpha$. For each system,
    \begin{compactenum}[i.]
        \item Determine the eigenvalues in terms of $\alpha$.
        \item Find the critical value(s) of $\alpha$ where the qualitative 
        nature of the phase portrait of the system changes.
        \item Make phase portraits for a value of $\alpha$ slightly below, and for another value slightly above, each critical value.
    \end{compactenum}

    \begin{parts}
        \begin{minipage}[t]{0.47\linewidth}
        \vspace{0pt}
            \part \begin{equation*}
                \vec{x}' = \begin{pmatrix}
                    \alpha & 1 \\ -1 & \alpha
                \end{pmatrix} \vec{x}
            \end{equation*} 

            \vspace{2in}

            \part \begin{equation*}
                \vec{x}' = \begin{pmatrix}
                    2 & -5 \\ \alpha & -2
                \end{pmatrix} \vec{x}
            \end{equation*} 

            \vspace{2in}
        \end{minipage}
        \begin{minipage}[t]{0.47\linewidth}
        \vspace{0pt}
            \part \begin{equation*}
                \vec{x}' = \begin{pmatrix}
                    -1 & \alpha \\ -1 & -1
                \end{pmatrix} \vec{x}
            \end{equation*} 

            \vspace{2in}

            \part \begin{equation*}
                \vec{x}' = \begin{pmatrix}
                    \alpha & 10 \\ -1 & -4
                \end{pmatrix} \vec{x}
            \end{equation*} 

            \vspace{2in}   
        \end{minipage}      
        
    \end{parts}

\end{questions} 

\end{document}