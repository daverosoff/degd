\documentclass[symmetric]{tufte-handout}
\usepackage{
    %graphicx,
    %wrapfig,
    fontspec,
    amsmath,
    amsthm,
    amssymb,
    latexsym,
    paralist,
    enumerate,
    epigraph,
    nicefrac,
    booktabs,
    xcolor,
}
\usepackage{rosoff}
\usepackage{tabularx,dcolumn}
\frenchspacing
\renewcommand\allcapsspacing[1]{{\addfontfeature{LetterSpace=15}#1}}
\renewcommand\smallcapsspacing[1]{{\addfontfeature{LetterSpace=10}#1}}
\linespread{1.05}
\setmainfont{Palatino Linotype}
\title{Mathematics 352 Differential Equations}
\date{Spring 2014}
\author{The College of Idaho}
\begin{document}
\maketitle

\newcommand{\officehours}{TBA}
%\newcommand{\officehours}{T~10–12; W~2–3; F~1–2}

\setlength{\epigraphwidth}{0.7\linewidth}
\renewcommand{\epigraphsize}{\normalsize}
\setlength{\epigraphrule}{0pt}
\setlength{\beforeepigraphskip}{0\baselineskip}
\setlength{\afterepigraphskip}{0\baselineskip}

\begin{fullwidth}
\epigraph{%
    I didn't discover curves; I only uncovered them.%
}{M\smallcaps{ae} W\smallcaps{est}}
\end{fullwidth}

\subsection{Quick Reference} \label{ssec:quickreference}
    \begin{minipage}[t]{0.44\linewidth}
        \vspace{0pt}
        Homework is assigned from the textbook and on WeBWorK. 
        You are free to work together on problems, but all work
        you submit must be your own. Weeklies may not be assigned
        every week, but you will have about a week to work on the
        problems. Projects will be presented during the final exam
        period.
    \end{minipage}\marginnote{% Office hours and contact info
        \begin{description}
            \item[Instructor]
                Dr.\ Dave Rosoff
            \item[Office]
                Boone Hall 102C
            \item[Office hours]
                MF~11:30–12:30, T~9–10, W~2:10–3:10 or by appointment
            \item[Email]
                \href{mailto:drosoff@collegeofidaho.edu}{drosoff@collegeofidaho.edu}
            \item[Website]
                \url{https://zeus.collegeofidaho.edu/academics/MathPhysics/courses/MAT-352/}
            \item[WeBWorK]
                \url{https://webwork.collegeofidaho.edu/webwork2/MAT352_S14}
            \item[Twitter]
                \texttt{@daverosoff}
        \end{description}
        } \hspace{1em}
    \begin{minipage}[t]{0.44\linewidth}
        \vspace{0pt}
        \centering\allcaps{Grading}\vspace*{1.0ex}
        \begin{tabular}{lcl}
            \toprule
            Tier & Weight & Date \\
            \midrule
            WeBWorK       & 0.10 & continual \\
            Workshops     & 0.05 & continual \\
            Weeklies      & 0.05 & continual \\  
            Quizzes       & 0.35 & 1.5x/week \\
            Project       & 0.06 & May 14 \\
            Midterm 1     & 0.13 & March 5 \\
            Midterm 2     & 0.13 & April 9 \\
            Midterm 3     & 0.23 & May 5 \\
            \bottomrule
        \end{tabular}
    \end{minipage} \hspace*{1em}

\subsection{Prerequisite}

The prerequisite for this course is MAT-251.\sidenote{Or equivalent.}

\subsection{Preface: Learning outcomes} \label{ssec:prefacelearningoutcomes}

    This course is designed to provide certain experiences, called “learning
    outcomes”, to students who successfully complete it. These outcomes are
    enumerated in the margin.%
    \sidenote{\smallcaps{Learning outcomes:}
        \begin{compactenum}
            \item \label{lo:recog} Recognize and describe fundamental ideas from course content
            described below.
            \item \label{lo:illust} Illustrate ideas from the course with examples.
            \item \label{lo:write} Improve their mathematical writing skills.
            \item \label{lo:preesh} Appreciate the historical context of differential equations 
            and their role in the physical sciences.
            \item \label{lo:develop} Plan, organize, and combine arguments to solve new problems, as
            appropriate.
            \item \label{lo:proc} Solve certain types of differential equations or initial value
            problems, or verify that a particular function is a solution.
            \item \label{lo:phenom} Recognize phenomena related to differential equations 
            (e.g., resonance or stability) and explain how they arise.
        \end{compactenum}    
    }
    I explicitly include these outcomes in the syllabus so that it is clear
    why I have chosen the various course components (each of which is
    described below.) Each learning outcome is addressed by one or more
    components of the course: quizzes, WeBWorK exercises, longer weekly problem sets, group research projects,
    presentations to the class, and exams.
    See the \emph{Grading} section below for more information.

\subsection{Introduction}

If you are considering a life as any kind of engineer, physical scientist, or
applied mathematician, life is differential equations. The laws of the
universe are, without exception, differential equations: problems involving
fluid flow, current in electric circuits, population dynamics, pricing of
financial derivatives,\footnote{Financial derivatives are unrelated to the
derivatives of calculus.} quantum mechanics, heat transfer, relativity, etc.,
are all formulated in this language. They are ubiquitous and of paramount
importance for the understanding of any chemical, physical, or economic 
phenomena. I hope this is reason enough for you to be interested.

\newthought{The recent history of scientific progress}, say the last 300–400
years, is characterized by a succession of \emph{models} of various phenomena.
These models tend to be increasingly mathematical with time. Classical
physics, for example, has been thoroughly mathematized since the time of
Newton (1642–1727), and quantum physics could not have been invented without
it, while the mathematization of biology is still in its nascent phase.

Mathematical models of phenomena that change in time (or space, or in regard
to any other measurable quantity) make essential use of derivatives. Without
derivatives, it is difficult to imagine a really serious formulation of
any phenomenon that is not the same in all places, at all times. The very idea
of changeability implies the necessity of \emph{rates of change}, because 
most of the phenomena of importance in life appear to be \emph{continuous} ones.
Moreover the ways in which these phenomena change are evidently also 
continuous.

A mathematical model can be something as simple as an equation. In fact, they
usually are. For example, balances remaining on outstanding student loans
can be modeled with exponential functions. Exactly what this means is 
somewhat nebulous, because no exponential function is equal to the actual
balance at all times (since real loans are neither paid continuously nor does
their interest accumulate continuously). Yet somehow the essential behavior
of the balance function is captured by the simpler exponential model.

Similarly, the level of the tide in the Bay of Fundy is not exactly given
by any sinusoidal function, but sinusoidal models of the tide and tides
around the world seem to be close enough to inform our understanding of these
phenomena.

These effective models arise from differential models of the underlying
rates. The tidal system in a narrow bay is described by just a few rates.
These rates, starting from physical first principles, are related in a 
way that is algebraically quite simple. The relationship of the rates—the
derivatives—is a differential equation, and the solutions of this equation
are functions describing possible tidal behavior.

Similarly, while most students have encountered the exponential model of
a loan or investment as a limiting phenomenon,%
    \sidenote{
        Continuous interest is the limit of ordinary interest as the length
        of the compounding period approaches 0.
    }
it is perhaps better understood in terms of arising from a specific differential
model. Let us suppose that the accumulation of interest is the only way in 
which money enters or leaves the account. Then, since interest is a constant 
multiple of principal,%
    \sidenote{
        The constant of proportionality is called the \emph{interest rate}.
    }
and the rate of increase of the account balance is identical with the accumulation
of interest, we must have $dP/dt = rP$, where $P$ is the principal and $r$ is
the interest rate.

\newthought{Here is a bit of mathematical} philosophy. Numbers and points are the same.
Ordinary numbers are points on a line; points in other spaces are different
kinds of numbers. If the space the points live in is a Euclidean space like
$\R^2$ or $\R^3$, these numbers are often called \emph{vectors}. You have
spent many years solving equations involving both constants and variables
whose values are ordinary numbers, and maybe a few weeks solving equations
involving vectors. Usually, we refer to this kind of work as \emph{algebra}.

Points, and therefore numbers, are zero-dimensional.\footnote{For points in 
the vector spaces $\R^2$ or $\R^3$, think of the heads
of the vectors, not the whole arrow.} They are pure locations. They do not
have extent of any kind. A one-dimensional number is called a \emph{curve}.%
    \sidenote{
        On the other hand, a vector (the whole arrow) is another kind of 
        one-dimensional number. Are vectors and 
        curves the same? Not really—but curves (at least differentiable ones) 
        have tangent vectors, so they seem to be related.
    }
You can also think of a curve as a family of points, as we often do in
calculus. These points are related—as are the members of any family. Often,
they are related by an equation defining the curve. Many algebraic problems
involve finding a special point on a curve: a special member of a family.

Differential equations also involve both constants and variables and need to
be solved. But here, the constants, variables, and solutions are all 
potentially one-dimensional: they are curves! The solutions are called the 
\emph{integral curves} of
the differential equation or equations. The solution of differential equations
is the understanding of the algebra of curves.%
    \sidenote{
        This is not a merely casual metaphor. The subject of \emph{differential
        geometry} involves a precise formulation of the algebra of curves (and
        higher-dimensional surfaces and hypersurfaces).
    }
Geometrically, the plane will be
partitioned into a bunch of related curves that fit together in a
\emph{family}—related, just as in algebra, by an equation (now, a differential
equation) defining the family. Many problems in differential equations involve
finding a special curve in a family.%
    \sidenote{
        Compare to the above description of solution of algebraic equations.
    }
These are called \emph{initial value
problems}. As in algebra, most equations cannot be solved. Equations that
admit exact solutions are very special. We will investigate some of the
elementary equations and classify them according to their form. We will also
discuss systems of linear differential equations, which are solved via 
linear-algebraic techniques involving matrices.

\subsection{Catalog description}

“A study of the solution and applications of ordinary differential equations
including systems of equations using matrix algebra.”

\subsection{Text}

The text is \emph{Elementary Differential Equations} by Boyce
and DiPrima, ninth edition. This book and its cousins%
    \sidenote{
        In particular, \emph{Elementary Differential Equations
        and Boundary Value Problems}, which is an acceptable
        substitute for our class.
    }
have been the standard
for two generations, so the world is awash in earlier editions (as well as
one much more expensive later edition). I don't recommend you use an edition
earlier than the eighth. If you have an earlier edition of the text than the
ninth, that is fine: but ascertaining the differences in the texts,
especially differently numbered or new homework problems, \emph{will be your
sole responsibility}. The same caveat applies to any international version.
Websites exist that catalog differences between versions.

\subsection{Grading}

Scores%
    \sidenote{\centering\smallcaps{Table of weights}
        \begin{tabular}{lc}
            \toprule
            Tier & Weight \\
            \midrule
            WeBWorK       & 0.10 \\
            Workshops     & 0.05 \\
            Weeklies      & 0.05 \\
            Quizzes       & 0.35 \\
            Project       & 0.06 \\
            Midterm 1     & 0.13 \\
            Midterm 2     & 0.13 \\
            Midterm 3     & 0.23 \\
            \bottomrule
        \end{tabular}
    }
are computed as a weighted average, with the weights indicated in the
margin. Descriptions of the various tiers follow. 
Observe that the weights sum to 1 = 100\%. The exact determination of letter
grades from these scores depends on the final distribution of scores in the
class, but you can expect a C for earning 73\% of the points, a C+ for
77\%, a B– for 80\%, and so on. I may adjust these cutoffs downward, but
I will not adjust them upward.

\subsection{Homework}

Both WeBWorK and traditional pencil-and-paper homework will be assigned.
WeBWorK exercises are shorter and more procedural, helping you assess your
own understanding of the ideas of the course. The weekly homework sets, 
which are longer, are more involved problems. Some problems 
may take you several hours, or require some work with a computer algebra
system such as Sage, Mathematica, or Matlab.%
    \sidenote{
        I believe that in fact it is possible to do all of the problems by
        hand, but some would be very laborious without computer assistance.
    }
My goal is as follows: every
student who successfully completes all of the homework problems and
\emph{understands all the solutions} should be able to earn an A in this
course. All quizzes and exams are designed with this in mind. Therefore, I
hope you will agree that it is in your very best interest to complete all of
the assigned work, regardless of whether it is turned in for credit. The
course is designed so that you will do best if you work at a modest but
constant pace throughout the term. Cramming might work too, but not as
well—and not as permanently, which is really the point.

\subsection{A note on written homework}

Even it you are not turning it in, it is important to pay attention to the
style of your writing and your presentation. Good mathematical writing is
essential for anyone who wishes to think clearly about mathematics—sloppy
writing \emph{invariably} reflects underlying sloppy thinking.%
    \sidenote{
        Since mathematics is one of the hardest things to think about,
        its careful study is the best training for proficiency in any
        kind of thought.
    }
The process of
making your ideas and reasoning \emph{clear, complete, and unambiguously
correct} is the greatest amplifier of mathematical power there is. Hence
your solutions should be composed in brilliant English prose.%
    \sidenote{
        This means conforming to accepted
        scientific usage, more or less correct grammar and spelling, and above all
        \emph{complete sentences}, sprinkled with equations here and there. Solutions
        in the popular “pile-of-equations” style are to be avoided and will not get
        much credit.
    }
You must explain what is happening as the action unfolds. You should also
avoid falling into a “two-column” format that you may have learned in a 
high-school geometry class. It is stilted, artificial, and not easier to read than
a pile of equations. Weave text and equations together for a gentle
presentation that doesn't leave the reader guessing.

\subsection{Quizzes}

Quizzes%
    \sidenote{
        Quizzes constitute the single largest component of the course grade.
        It is very difficult to get a high grade in the class without getting
        a high quiz grade. Since quizzes are so frequent, this is eminently
        possible, but it is extremely advisable to stay current with reading
        and WeBWorK in order to effect it.
    }
will be given frequently to help you make sure you are staying on top
of the material. Quiz problems will come directly, or nearly so, from the
assigned daily WeBWorK problems. Quizzes begin promptly at 8:00 and cannot be
made up.

\subsection{Workshops}

Most of you are familiar with the workshop format%
    \sidenote{
        Workshops are generally marked for full credit if it appears
        you made an honest effort to complete them. 
    }
from Calculus~III. Workshops
are primarily in-class activities (although some will need to be finished at home)
in which you will work through some aspects of a problem with other students in
the class. Along with the homework, the workshops are the heart of the course.

\subsection{Exam dates}

As specified above, you will take three hourlong midterm exams. Let me know
\emph{immediately} if you foresee a conflict. See below also for information
regarding make-up exams.

\begin{compactitem}
   \item Exam~1 (tentative): Wednesday, March~5
   \item Exam~2 (tentative): Wednesday, April~9
   \item Exam~3 (tentative): Wednesday, May~5 (week 12, not finals week)%
        \sidenote{
            Contrary to occasionally expressed opinion, the last week of
            class is very much alive.
        }
\end{compactitem}
%
\subsection{Make-ups}

I will only consider make-up exams with a \emph{documented, compelling reason}
and sufficient (two weeks is always enough) notice; otherwise, remaining exams
will be reweighted.  Quizzes can not be made up.

\subsection{Academic integrity}

Students are expected to complete all graded work in accordance with the
College Honor Code. Plagiarism, cheating, or borrowing without proper credit
will not be tolerated.  Violations of academic honesty can result in loss of
credit on an assignment, failure on an exam, or failure in the course. A
referral may be made to the Vice President for Academic Affairs for all
parties involved in academic dishonesty.

\subsection{A note on studying math}

This class is all about practice. Learning to recognize the different types of
equations and quickly and accurately recall the procedures needed to solve
them is a matter of rote technique. If you have practiced enough, you will
remember how to do it.\footnote{This is the definition of “enough”.} If not,
something else may happen. Of course, there are deep ideas lurking beneath the
computational techniques. Awareness and understanding of these ideas also
comes with practice.

\subsection{Disability statement}

Students with documented disabilities as addressed by the Americans With
Disabilities Act and who need any test or course materials to be furnished in
an alternative format should notify me immediately (during the first week of
class).  Reasonable efforts will be made to accommodate the needs of such
students. 

\dwrspace{1}

\begin{fullwidth}
    \centering {\Huge \allcaps{Good luck this semester!}}
\end{fullwidth}

\end{document}
